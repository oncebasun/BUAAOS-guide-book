%----------------------------------------------------------------------------------------
%	PACKAGES AND OTHER DOCUMENT CONFIGURATIONS
%----------------------------------------------------------------------------------------
\documentclass[11pt, a4paper,oneside]{book}

\input{title-page}

\usepackage{graphicx} % Required for including pictures
\graphicspath{{Images/}} % Specifies the directory where pictures are stored



%----------------------------------------------------------------------------------------
%       Localization
%----------------------------------------------------------------------------------------
\usepackage[UTF8,adobefonts]{ctex}
\usepackage{array, booktabs}
\usepackage{graphicx}
\usepackage[x11names]{xcolor}
\usepackage{colortbl}
\usepackage{fontspec}
\newcommand{\foo}{\color{baseD}\makebox[0pt]{\textbullet}\hskip-0.5pt\vrule width 1pt\hspace{\labelsep}}

%\setmainfont[Boldont=WenQuanYi Micro Hei]{AR PL SungtiL GB}
%\setsansfont[BoldFont=WenQuanYi Micro Hei]{AR PL KaitiM GB}
%\setmonofont{DejaVu Sans Mono}

%Please uncomment the following three lines, when you want to compile on Mac OSX. Comment out the following three lines before you push to Travis-CI
\setCJKmainfont[BoldFont={AdobeHeitiStd-Regular},ItalicFont={AdobeKaitiStd-Regular}]{AdobeSongStd-Light}
\setCJKsansfont[BoldFont={AdobeHeitiStd-Regular}]{AdobeKaitiStd-Regular}
\setCJKmonofont{AdobeFangsongStd-Regular}

%\XeTeXlinebreaklocale "zh"
%\XeTeXlinebreakskip = 0pt plus 1pt minus 0.1pt

\usepackage[top=1in,bottom=1in,left=1.25in,right=1.25in]{geometry}
%\linespread{1.2}

\usepackage[Glenn]{fncychap}

\usepackage{fancyhdr}

%----------------------------------------------------------------------------------------
%       Useful Packages
%----------------------------------------------------------------------------------------
\usepackage{color}
\usepackage{url}
\usepackage[colorlinks, linkcolor=black,anchorcolor=black, citecolor=black]{hyperref}

\usepackage{xcolor} % Required for specifying colors by name
\definecolor{ocre}{RGB}{243,102,25} % Define the orange color used for highlighting throughout the book

% BASE16
\definecolor{base0}{HTML}{181818}
\definecolor{base1}{HTML}{282828}
\definecolor{base2}{HTML}{383838}
\definecolor{base3}{HTML}{585858}
\definecolor{base4}{HTML}{B8B8B8}
\definecolor{base5}{HTML}{D8D8D8}
\definecolor{base6}{HTML}{E8E8E8}
\definecolor{base7}{HTML}{F8F8F8}
\definecolor{base8}{HTML}{AB4642}
\definecolor{base9}{HTML}{DC9656}
\definecolor{baseA}{HTML}{F7CA88}
\definecolor{baseB}{HTML}{A1B56C}
\definecolor{baseC}{HTML}{86C1B9}
\definecolor{baseD}{HTML}{7CAFC2}
\definecolor{baseE}{HTML}{BA8BAF}
\definecolor{baseF}{HTML}{A16946}
\definecolor{Gray}{HTML}{CCCCCC}
\definecolor{linkcolor}{HTML}{EC008C}
\definecolor{codecolorpink}{HTML}{CC00FF}
\definecolor{NoteColorFont}{HTML}{6D727D}
\definecolor{NoteColorLine}{HTML}{C3CAD9}
\definecolor{ExeColorFont}{HTML}{FF9900}
\definecolor{ExeColorLine}{HTML}{FFF678}
\definecolor{ExeColorBack}{HTML}{FFFFCC}
\definecolor{ThinkColorFont}{HTML}{629D81}
\definecolor{ThinkColorLine}{HTML}{93E87D}
\definecolor{ThinkColorBack}{HTML}{C1FA9B}

\usepackage{amsmath,amsfonts,amssymb,amsthm} % For math equations, theorems, symbols, etc
\usepackage{booktabs} % For tables
\usepackage{tabularx}
\usepackage{multirow} % for multiple row tables.

%----------------------------------------------------------------------------------------
%       Some Extra Definitions
%----------------------------------------------------------------------------------------

\RequirePackage[framemethod=default]{mdframed} % Required for creating the theorem, definition, exercise and corollary boxes

% Exercise box
\newmdenv[skipabove=10pt,
skipbelow=10pt,
rightline=false,
leftline=true,
topline=false,
bottomline=false,
backgroundcolor=ExeColorBack,
linecolor=ExeColorLine,
innerleftmargin=5pt,
innerrightmargin=5pt,
innertopmargin=5pt,
innerbottommargin=5pt,
leftmargin=0cm,
rightmargin=0cm,
linewidth=12pt]{eBox}

% Thinking box
\newmdenv[skipabove=10pt,
skipbelow=10pt,
rightline=false,
leftline=true,
topline=false,
bottomline=false,
backgroundcolor=ThinkColorBack!30,
linecolor=ThinkColorLine,
innerleftmargin=5pt,
innerrightmargin=5pt,
innertopmargin=5pt,
innerbottommargin=5pt,
leftmargin=0cm,
rightmargin=0cm,
linewidth=12pt]{tBox}

% Note box
\newmdenv[skipabove=10pt,
skipbelow=10pt,
rightline=false,
leftline=true,
topline=false,
bottomline=false,
backgroundcolor=NoteColorLine!15,
linecolor=NoteColorLine,
innerleftmargin=5pt,
innerrightmargin=5pt,
innertopmargin=5pt,
innerbottommargin=5pt,
leftmargin=0cm,
rightmargin=0cm,
linewidth=12pt]{nBox}

% Boxed/framed environments
\newtheoremstyle{ocrenumbox}% % Theorem style name
{0pt}% Space above
{0pt}% Space below
{\normalfont}% % Body font
{}% Indent amount
{\small\bf\sffamily\color{ExeColorFont}}% % Theorem head font
{\;}% Punctuation after theorem head
{0.25em}% Space after theorem head
{\small\sffamily\color{ExeColorFont}\thmname{#1}\nobreakspace\thmnumber{#2}% Theorem text (e.g. Exercise 2.1)
\thmnote{\nobreakspace\the\thm@notefont\sffamily\bfseries\color{black}---\nobreakspace#3.}} % Optional theorem note
\renewcommand{\qedsymbol}{$\blacksquare$}% Optional qed square

\newtheoremstyle{purplenumbox}% % Theorem style name
{0pt}% Space above
{0pt}% Space below
{\normalfont}% % Body font
{}% Indent amount
{\small\bf\sffamily\color{ThinkColorFont}}% % Theorem head font
{\;}% Punctuation after theorem head
{0.25em}% Space after theorem head
{\small\sffamily\color{ThinkColorFont}\thmname{#1}\nobreakspace\thmnumber{#2}
% Theorem text (e.g. Thinking 2.1)
\thmnote{\nobreakspace\the\thm@notefont\sffamily\bfseries\color{black}---\nobreakspace#3.}} % Optional theorem note
\renewcommand{\qedsymbol}{$\blacksquare$}% Optional qed square

\newtheoremstyle{blackbox} % Theorem style name
{0pt}% Space above
{0pt}% Space below
{\normalfont}% Body font
{}% Indent amount
{\small\bf\sffamily}% Theorem head font
{\;}% Punctuation after theorem head
{0.25em}% Space after theorem head
{\small\sffamily\color{NoteColorFont}\thmname{#1}\nobreakspace\thmnumber{#2}
% Theorem text (e.g. Theorem 2.1)
\thmnote{\nobreakspace\the\thm@notefont\sffamily\bfseries---\nobreakspace#3.}}% Optional theorem note

% Defines the theorem text style for each type of theorem to one of the three styles above
\theoremstyle{ocrenumbox}
\newtheorem{exerciseT}{Exercise}[chapter]
\theoremstyle{purplenumbox}
\newtheorem{thinkingT}{Thinking}[chapter]
\theoremstyle{blackbox}
\newtheorem{noteT}{Note}[section]

\newenvironment{exercise}{\begin{eBox}\begin{exerciseT}}{\hfill{\color{ExeColorFont}\tiny\ensuremath{\blacksquare}}\end{exerciseT}\end{eBox}}
\newenvironment{thinking}{\begin{tBox}\begin{thinkingT}}{\hfill{\color{ThinkColorFont}\tiny\ensuremath{\blacksquare}}\end{thinkingT}\end{tBox}}
\newenvironment{note}{\begin{nBox}\begin{noteT}}{\end{noteT}\end{nBox}}

%----------------------------------------------------------------------------------------
%       Code Environment
%----------------------------------------------------------------------------------------
\usepackage{minted}
\usemintedstyle{manni}

% code box
\newmdenv[backgroundcolor=base7,
linecolor=baseD,
bottomline=false,
leftline=true,
rightline=false,
topline=false,
linewidth=2pt,
leftmargin=13pt]{pcodeBox}

\renewcommand{\theFancyVerbLine}{
  \sffamily
  \textcolor{baseB}{\arabic{FancyVerbLine}
  }
}

\usepackage{caption}

%\captionsetup{type=codeCaption}
\newenvironment{codeBox}{\begin{pcodeBox}\fontsize{9pt}{9pt}}{\end{pcodeBox}}
\newenvironment{codeBoxWithCaption}[1]{\begin{pcodeBox}[frametitle={\captionof{listing}{#1}\color{base6}\rule{\textwidth}{0.7pt}}]\fontsize{9pt}{9pt}}{\end{pcodeBox}}

\BeforeBeginEnvironment{minted}{\begin{codeBox}}
\AfterEndEnvironment{minted}{\end{codeBox}}

%----------------------------------------------------------------------------------------
%       Lists
%----------------------------------------------------------------------------------------
\usepackage{enumitem}
\setlist[description]{labelindent=22pt} 

%----------------------------------------------------------------------------------------
%       Main Body
%----------------------------------------------------------------------------------------
\begin{document}

\pagestyle{empty} % Removes page numbers
\titleGP % This command includes the title page

\frontmatter
\pagestyle{plain}

\input{preface/editors}
\input{preface/teacher}

\frontmatter
\tableofcontents

\mainmatter
\pagestyle{fancy}
%\input{chapters/0-environment}
\input{chapters/1-start}
\chapter{内存管理}

\section{实验目的}
  \begin{enumerate}
    \item 了解MIPS内存映射布局
    \item 掌握使用空闲链表的管理物理内存的方法
    \item 建立页表,实现分页式虚存管理
    \item 实现内存分配和释放的函数
  \end{enumerate}

本次实验中,我们需要掌握MIPS页式内存管理机制,需要使用一些数据结构来记录内存的使用情况,并实现内存分配
和释放的相关函数,完成物理内存管理和虚拟内存管理。

\section{MMU和TLB}

首先介绍两个与内存管理有关的概念:MMU和TLB。

\begin{note}
MMU,即内存管理单元(memory‐management unit)。MMU 是 CPU 中用来管理虚拟存储器、物理存储器的
控制线路,同时也负责虚拟地址映射为物理地址,以及提供硬件机制的内存访问授权。
\end{note}

\begin{note}
TLB,即后备缓冲(translation lookaside buffer)。TLB 是将程序使用的地址(虚拟地址)翻译成物理地址
(访问内存的地址)的硬件。
\end{note}

\section{MIPS虚存映射布局}

32位的MIPS CPU最大寻址空间为4GB(2\^32字节),这4GB虚存空间被划分为四个部分:

\begin{enumerate}
  \item kuseg (TLB-mapped cacheable user space, 0x00000000 ~ 0x7fffffff):
  这一段是用户模式下可用的地址,大小为2G,也就是MIPS约定的用户内存空间。需要通过MMU进行虚拟地址到物理
  地址的转换。
  \item kseg0 (direct-mapped cached kernel space, 0x80000000 ~ 0x9fffffff):
  这一段是内核地址,其内存虚存地址到物理内存地址的映射转换不通过MMU,使用时只需要将地址的最高位清零
  (\& 0x7fffffff),
  这些地址就被转换为物理地址。也就是说,这段逻辑地址被连续地映射到物理内存的低端512M空间。对这段地址
  的存取都会通过高速缓存(cached)。通常在没有MMU的系统中,这段空间用于存放大多数程序和数据。对于有
  MMU 的系统,操作系统的内核会存放在这个区域。
  \item kseg1 (direct-mapped uncached kernel space, 0xa0000000 ~ 0xbfffffff):
  与kseg0类似,这段地址也是内核地址,将虚拟地址的高 3 位清零(\& 0x1fffffff),就可以转换到物理地址,
  这段逻辑地址也是被连续地映射到物理内存的低端512M空间。但是 kseg1 不使用缓存(uncached),访问速度比较慢,
  但对硬件I/O寄存器来说,也就不存在Cache一致性的问题了,这段内存通常被映射到I/O寄存器,用来实现对外设的访问。
  \item kseg2 (TLB-mapped cacheable kernel space, 0xc0000000 ~ 0xffffffff):
  这段地址只能在内核态下使用,并且需要 MMU 的转换。
\end{enumerate}

\section{内存管理}

在第一实验中,我们将内核加载到内存中的 kseg0 区域(0x80010000),成功启动并跳转到 init/main.c 中的
 main 函数开始运行,现在我们需要在 main 函数中调用定义在 init/init.c 中的 mips\_init() 函数,并
进一步通过

\begin{enumerate}
  \item \mintinline{c}|mips_detect_memory()|;
  \item \mintinline{c}|mips_vm_init()|;
  \item \mintinline{c}|page_init()|;
\end{enumerate}

这三个函数来实现内存管理的相关数据结构的初始化。

\section{物理内存管理}

\subsection{内存控制块}

在MIPS CPU 中,地址转换以4KB 大小为单位,称为页。我们使用 Page 结构体来作为记录一页内存的相关信息
的数据结构:

\begin{minted}[linenos]{c}
typedef LIST_ENTRY(Page) Page_LIST_entry_t;

struct Page {
    Page_LIST_entry_t pp_link;  /* free list link */
    u_short pp_ref;
};
\end{minted}

其中,\mintinline{c}|pp_ref| 用来记录这一物理页面的引用次数,\mintinline{c}|pp_link| 是当前节点
指向链表中下一个节点的指针,其类型为 \mintinline{c}|LIST_ENTRY(Page)| 。我们在 include/queue.h
中定义了一系列的宏函数来简化对链表的操作。请阅读这些宏函数的代码,理解它们的原理和巧妙之处。

\begin{thinking}\label{think-do_while}
我们注意到我们把宏函数的函数体写成了 \mintinline{c}|do{/* ... */}while(0)| 的形式,而不是
仅仅写成形如
\mintinline{c}|{ /* ... */ }| 的语句块,这样的写法好处是什么?
\end{thinking}

在 include/pmap.h 中,我们使用 LIST\_HEAD 宏来定义了一个结构体类型 Page\_list ,在 mm/pmap.c
中,创建了一个该类型的变量 page\_free\_list 来以链表的形式表示所有的空闲物理内存:

\begin{minted}[linenos]{c}
LIST_HEAD(Page_list, Page);

static struct Page_list page_free_list; /* Free list of physical pages */
\end{minted}

\subsection{内存分配和释放}

首先,我们需要注意在 mm/pmap.c 中定义的与内存相关的全局变量:

\begin{minted}[linenos]{c}
u_long maxpa;            /* Maximum physical address */
u_long npage;            /* Amount of memory(in pages) */
u_long basemem;          /* Amount of base memory(in bytes) */
u_long extmem;           /* Amount of extended memory(in bytes) */
\end{minted}

\begin{exercise}
我们需要在 mips\_detect\_memory() 函数中初始化这几个全局变量,以确定内核可用的物理内存的大小和范围。
根据代码注释中的提示,完成 mips\_detect\_memory() 函数。
\end{exercise}

在操作系统刚刚启动时,我们还没有建立起有效的数据结构来管理所有的物理内存,因此, 出于最基本的
内存管理的需求,我们需要实现一个函数来分配指定字节的物理内存。这一功能由 mm/pmap.c 中的
alloc函数来实现。

\begin{minted}[linenos]{c}
static void *alloc(u_int n, u_int align, int clear);
\end{minted}

alloc 函数能够按照参数 align 进行对齐,然后分配 n 字节大小的物理内存,并根据参数 clear 的设定决
定是否将新分配的内存全部清零,并最终返回新分配的内存的首地址。

有了分配物理内存的功能后,接下来我们需要给操作系统内核必须的数据结构 -- 页表(pgdir)、内存控制块
数组(pages)和进程控制块数组(envs)分配所需的物理内存。mips\_vm\_init() 函数实现了这一功能,
并且完成了相关的虚拟内存与物理内存之间的映射。

完成上述工作后,我们便可以通过在 mips\_init() 函数中调用 page\_init() 函数将余下的物理内存块加
入到空闲链表中。

\begin{exercise}
完成 page\_init 函数,使用 include/queue.h 中定义的宏函数将未分配的物理页加入到空闲链表
page\_free\_list 中去。思考如何区分已分配的内存块和未分配的内存块,并注意内核可用的物理内存上限。
\end{exercise}

有了记录物理内存使用情况的链表之后,我们就可以不再像之前的 alloc 函数那样按字节为单位进行内存
的分配,而是可以以页为单位 进行物理内存的分配与释放。page\_alloc 函数用来从空闲链表中分配一页
物理内存,而 page\_free 函数则用于将一页之前分配的内存重新加入到空闲链表中。

\begin{exercise}
完成 mm/pmap.c 中的 page\_alloc 和 page\_free 函数,基于空闲内存链表 page\_free\_list ,以页
为单位进行物理内存的管理。
\end{exercise}

至此,我们的内核已经能够按照分页的方式对物理内存进行管理。

\section{虚拟内存管理}

我们通过建立两级页表来进行虚拟内存的管理,在此基础上,我们将实现在页表中根据虚拟地址在页表中查找
对应的物理地址,将一段虚存地址映射到一段的物理地址,然后实现虚存的管理与释放,最后为内核建立起一
套虚存管理系统。

\subsection{两级页表机制}

我们的操作系统内核采取二级页表结构,如图\ref{lab2-pic-1.png}所示:

\begin{figure}[htbp]
  \centering
  \includegraphics[width=12cm]{lab2-pic-1.png}
  \caption{二级页表结构示意图}\label{lab2-pic-1.png}
\end{figure}

第一级表称为页目录(page directory),一共1024个页目录项,每个页目录项32位(4 Byte),页目录项存储
的值为其对应的二级页表入口的物理地址。整个页目录存放在一个页面(4KB)中,也就是我们在 mips\_vm\_init
函数中为其分配了相应的物理内存。第二级表称为页表(page table),每一张页表有1024个页表项,每个页表
项32位(4 Byte),页表项存储的是对应页面的页框号(20位)以及标志位(12位)。每张页表占用一个页面大小
(4KB)的内存空间。

对于一个32位的虚存地址,其31-22位表示的是页目录项的索引,21-12位表示的是页表项的索引,11-0位表示的
是该地址在该页面内的偏移。

\subsection{地址转换}

对于操作系统来说,虚拟地址与物理地址之间的转换是内存管理中非常重要的内容。在这一部分,我们将详细探讨
咱们的内核是如何进行地址转换的。

首先从较为简单的形式开始。在前面的实验中,我们通过设置 lds 文件让操作系统内核加载到内存的 0x80010000
位置,在上文我们对 MIPS 存储器映射布局的介绍中我们知道,这一地址对应的是 kseg0 区域,这一部分的地址
转换不通过 MMU 进行。我们也称这一部分虚拟地址为内核虚拟地址。从虚拟地址到物理地址的转换只需要清掉
最高位的零即可,反过来,将对应范围内的物理地址转换到内核虚拟地址,也只需要将最高位设置为1即可。我们在
 include/mmu.h 中定义了 PADDR 和 KADDR 两个宏来实现这一功能:

\begin{minted}[linenos]{c}
// translates from kernel virtual address to physical address.
#define PADDR(kva)            \
  ({                \
    u_long a = (u_long) (kva);        \
    if (a < ULIM)         \
      panic("PADDR called with invalid kva %08lx", a);\
    a - ULIM;           \
  })

// translates from physical address to kernel virtual address.
#define KADDR(pa)           \
  ({                \
    u_long ppn = PPN(pa);         \
    if (ppn >= npage)         \
      panic("KADDR called with invalid pa %08lx", (u_long)pa);\
    (pa) + ULIM;          \
  })
\end{minted}

在 PADDR 中,我们使用了一个宏 ULIM ,这个宏定义在 include/mmu.h 中,其值为 0x80000000。对于小于
 0x80000000 的虚拟地址值,显然不可能是内核区域的虚拟地址。在 KADDR 中,一个合理的物理地址的物理
页框号显然不能大于我们在 mm/pmap.c 中所定义的物理内存总页数 npage 的值。

接下来,我们讨论如何通过二级页表进行虚拟地址到物理地址的转换。

首先,我们可以通过 PDX(va) 来获得一个虚拟地址对应的页目录索引,然后直接以凭借索引在页目录中得到对应的
二级页表的基址(物理地址),然后把这个物理地址转化为内核虚拟地址(KADDR),之后,通过 PTX(va) 获得这个
虚存地址对应的页表索引,然后就可以从页表中得到对应的页面的物理地址。整个转换的过程如 图\ref{lab2-pic-2.png}
所示:

\begin{figure}[htbp]
  \centering
  \includegraphics[width=12cm]{lab2-pic-2.png}
  \caption{地址转换过程}\label{lab2-pic-2.png}
\end{figure}

\subsection{页目录自映射}

上文中我们讲二级页表结构的时侯提到,要映射整个4G地址空间,一共需要1024个页表和1个页目录的。一个页表占用 4KB
空间,页目录也占用 4KB 空间,也就是说,整个二级页表结构将占用 4MB+4KB 的存储空间,1024*4KB+1*4KB=4MB+4KB。

在 include/mmu.h 中的内存布局图里有这样的内容:

\begin{minted}[linenos]{c}
/**
 *      ULIM     -----> +----------------------------+------------0x8000 0000-------
 *                      |         User VPT           |     PDMAP                /|\
 *      UVPT     -----> +----------------------------+------------0x7fc0 0000    |
 */
\end{minted}

不难计算出 UVPT(0x7fc00000) 到 ULIM(0x80000000) 之间的空间只有 4MB ,这一区域就是进程的
页表的位置,于是我们不禁想问:页目录所占用的 4KB 内存空间在哪儿?

\textbf{答案就在于页目录的自映射机制!}

如果页表和页目录没有被映射到进程的地址空间中,而一个进程的4GB地址空间又都映射了物理内存的话,那么就确实需要
1024个物理页(4MB)来存放页表,和另外1个物理页(4KB)来存放页目录,也就是需要(4M+4K)的物理内存。但是页表也被映射
到了进程的地址空间中,也就意味着 1024 个页表中,有一个页表所对应的 4M 空间正是这 1024 个页表所占用的 4M 内存,
这个页表的 1024 的页表项存储了 1024 个物理地址,分别是 1024 个页表的物理地址。而在二级页表结构中,页目录
对应着二级页表,1024 个页目录项存储的也是全部 1024 个页表的物理地址。也就是说,一个页表的内容和页目录的
内容是完全一样的,正是这种完全相同,使得将 1024 个页表加 1 个页目录映射到地址空间只需要 4M 的地址空间,
\textbf{其中的一个页表和页目录完全重合了}。

接下来我们会想到这样一个问题:那么,这个与页目录重合的页表,也就是页目录究竟在哪儿呢?

我们知道,这 4M 空间的起始位置也就是第一个二级页表对应着页目录的第一个页目录项,同时,由于 1M 个页表项和 4G
地址空间是线性映射,不难算出 0x7fc00000 这一地址对应的应该是第 (0x7fc00000 >> 12) 个页表项,这个页表项
也就是第一个页目录项。一个页表项 32 位,占用 4 个字节的内存,因此,其相对于页表起始地址 0x7fc000000 的
偏移为 (0x7fc00000 >> 12) * 4 = 0x1ff000 ,于是得到地址为 0x7fc00000 + 0x1ff000 = 0x7fdff000 。也就是
说,页目录的虚存地址为 0x7fdff000。

\begin{thinking}\label{think-windows_pde_addr}
了解了二级页表页目录自映射的原理之后,我们知道,Win2k内核的虚存管理也是采用了二级页表的形式,其页表所占的 4M
空间对应的虚存起始地址为 0xC0000000,那么,它的页目录的起始地址是多少呢?
\end{thinking}

\subsection{创建页表}

将虚拟地址转换为物理地址的过程中,如果这个虚拟地址所对应的二级页表不存在,有时,我们可能需要为这个
虚拟地址创建一个新的页表。我们需要申请一页物理内存,然后将他的物理地址赋值给对应的页目录项,最后设置
好页目录项的权限位即可。

我们的内核在 mm/pmap.c 中分定义了 boot\_pgdir\_walk 和 pgdir\_walk 两个函数来实现地址转换和
页表的创建,这两个函数的区别仅仅在于当虚存地址所对应的页表的页面不存在的时候,分配策略的不同和使用的
内存分配函数的不同。前者用于内核刚刚启动的时候,这一部分内存通常用于存放内存控制块和进程控制块等数据
结构,相关的页表和内存映射必须建立,否则操作系统内核无法正常执行后续的工作。然而用户程序的内存
申请却不是必须满足的,当可用的物理内存不足时,内存分配允许出现错误。boot\_pgdir\_walk 被调用时,
还没有建立起空闲链表来管理物理内存,因此,直接使用 alloc 函数以字节为单位进行物理内存的分配,而
 pgdir\_walk 则在空闲链表初始化之后发挥功能,因此,直接使用 page\_alloc 函数从空闲链表中
以页为单位进行内存的申请。

上文中介绍了页目录自映射的相关知识,我们了解到页目录其实也是一个页表。我们知道,在咱们实验使用的内核中,
一个页表指向的物理页面存在的标志是页表项存储的值的 PTE\_V 标志位被置为 1 。因此,在将页表的物理地址赋值
给 页目录项时,我们还为页目录项设置权限位。

\begin{exercise}
完成 mm/pmap.c 中的 boot\_pgdir\_walk 和 pgdir\_walk 函数,实现虚拟地址到物理地址的转换以及创建页表的功能。
\end{exercise}

\subsection{地址映射}

\begin{exercise}
实现 mm/pmap.c 中的 boot\_map\_segment 函数,实现将制定的物理内存与虚拟内存建立起映射的功能。
\end{exercise}

\subsection{page insert and page remove}

\begin{minted}[linenos]{c}
// Overview:
//   Map the physical page 'pp' at virtual address 'va'.
//   The permissions (the low 12 bits) of the page table entry
//   should be set to 'perm|PTE_V'.
//
// Post-Condition:
//   Return 0 on success
//   Return -E_NO_MEM, if page table couldn't be allocated
//
// Hint:
//   If there is already a page mapped at `va`, call page_remove()
//   to release this mapping.The `pp_ref` should be incremented
//   if the insertion succeeds.
int
page_insert(Pde *pgdir, struct Page *pp, u_long va, u_int perm)
{
	u_int PERM;
	Pte *pgtable_entry;
	PERM = perm | PTE_V;

	/* Step 1: Get corresponding page table entry. */
	pgdir_walk(pgdir, va, 0, &pgtable_entry);

	if (pgtable_entry != 0 && (*pgtable_entry & PTE_V) != 0) {
		if (pa2page(*pgtable_entry) != pp) {
			page_remove(pgdir, va);
		}
		else{
			tlb_invalidate(pgdir, va);
			*pgtable_entry = (page2pa(pp) | PERM);
			return 0;
		}
	}

	/* Step 2: Update TLB. */
	tlb_invalidate(pgdir, va);

	/* Step 3: Do check, re-get page table entry to validate
	 *  the insertion. */
	if (pgdir_walk(pgdir, va, 1, &pgtable_entry) != 0) {
		return -E_NO_MEM;    // panic ("page insert failed .\n");
	}

	*pgtable_entry = (page2pa(pp) | PERM);
	pp->pp_ref++;
	return 0;
}
\end{minted}

这个函数虽然我们并没有填,但是\textbf{非常重要}!这个函数将在\textbf{lab3和lab4}中被反复用到,这个函数
将va虚拟地址和其要对应的物理页pp的映射关系以perm的权限设置加入页目录。我们大概讲一下函数的执行流程与执行要点。

\textbf{流程大致如下:}先判断va是否有对应的页表项:如果页表项有效(或者叫va是否已经有了映射的物理地址)
的话,则去判断这个物理地址是不是我们要插入的那个物理地址,如果不是,那么就把该物理地址移除掉;
如果是的话,则修改权限,放到tlb中。

\textbf{有一个值得指出的要点}:我们能看到,只要对页表的内容修改,都必须tlb\_invalidate 来让tlb更新,否则后面紧接着对内存的访问很有可能出错。

可以说tlb\_invalidate 函数是它的一个核心子函数,这个函数实际上又是由tlb\_out 汇编函数组成的。

\begin{codeBoxWithCaption}{TLB汇编函数\label{code:tlb_out.S}}
  \inputminted[linenos]{gas}{codes/tlb_out.S}
\end{codeBoxWithCaption}

这个汇编函数相对其他汇编函数来说相对简单,那么留给你一个思考问题.

\begin{thinking}\label{think-tlb}
思考一下tlb\_out 汇编函数,结合代码阐述一下跳转到\textbf{NOFOUND}的流程?
\end{thinking}

\section{正确结果展示}

实验二做完之后,正确的结果应该是这样的:

\begin{minted}[linenos]{c}
main.c: main is start ...
init.c: mips_init() is called
Physical memory: 65536K available, base = 65536K, extended = 0K
to memory 80401000 for struct page directory.
to memory 80431000 for struct Pages.
mips_vm_init:boot_pgdir is 80400000
pmap.c:  mips vm init success
start page_insert
va2pa(boot_pgdir, 0x0) is 3ffe000
page2pa(pp1) is 3ffe000
pp2->pp_ref 0
end page_insert
page_check() succeeded!
panic at init.c:55: ^^^^^^^^^^^^^^^^^^^^^^^^^^^^^^^^^^^^^
\end{minted}

最后一行的数字 \textbf{55} 是不固定的。

\section{实验思考}

\begin{itemize}
\item \hyperref[think-do_while]{\textbf{\textcolor{baseB}{思考-使用do-while(0)语句的好处}}}
\item \hyperref[think-windows_pde_addr]{\textbf{\textcolor{baseB}{思考-自映射机制页目录地址的计算}}}
\item \hyperref[think-tlb]{\textbf{\textcolor{baseB}{思考-NOFOUND的奥妙}}}
\end{itemize}

\section{挑战性任务}

我们的虚存管理系统分配和回收页面粒度的内存,问题是如果我们需要支持真正的I/O设备,需要分配超过4KB的物理连续的内存;如果我们需要用户态使用,为了提高CPU效率,我们需要分配大小超过4MB的超级页。修改虚存管理,使得可以一次申请2的指数大小的内存,你可以自行设置大小的上限。确保,在需要的时候,将大的存储单元分为小的存储单元;如果可能,将小的存储单元合并为大的存储单元。

\chapter{进程与异常}

\section{实验目的}
  \begin{enumerate}
    \item 创建一个进程并成功运行
    \item 实现时钟中断,通过时钟中断内核可以再次获得执行权
    \item 实现进程调度,创建两个进程,并且通过时钟中断切换进程执行
  \end{enumerate}

在本次实验中你将运行一个用户模式的进程。你需要使用数据结构进程控制块 Env 来跟踪用户进程。
通过建立一个简单的用户进程,加载一个程序镜像到进程控制块中,并让它运行起来。
同时,你的MIPS内核将拥有处理异常的能力。

\section{进程}

\begin{note}
进程既是基本的分配单元,也是基本的执行单元。第一,进程是一个实体。每一个进程都有它自己的地址空间,
一般情况下,包括代码段、数据段和堆栈。第二,进程是一个“执行中的程序”。
程序是一个没有生命的实体,只有处理器赋予程序生命时,它才能成为一个活动的实体,我们称其为进程。
\end{note}

\subsection{进程控制块}

这次实验是关于进程和异常的,那么我们首先来结合代码看看进程控制块是个什么东西。

进程控制块(PCB)是系统为了管理进程设置的一个专门的数据结构,用它来记录进程的外部特征,描述进程的运动变化过程。
系统利用PCB来控制和管理进程,所以\textbf{PCB是系统感知进程存在的唯一标志。进程与PCB是一一对应的。}
通常PCB应包含如下一些信息:

\begin{codeBoxWithCaption}{进程控制块\label{code:process_of_env.h}}
  \inputminted[linenos]{c}{codes/process_of_env.h}
\end{codeBoxWithCaption}

为了集中你的注意力在关键的地方,我们暂时不介绍其他实验所需介绍的内容。
下面是\textbf{lab3}中需要用到的这些域的一些简单说明:

\begin{itemize}
  \item env\_tf : Trapframe 结构体的定义在\textbf{include/trap.h} 中,env\_tf 的作用就是在进程因为时间片用光不再运行时,将其当时的进程上下文环境保存在env\_tf 变量中。当从用户模式切换到内核模式时,内核也会保存进程上下文,因此进程返回时上下文可以从中恢复。
  \item env\_link : env\_link 的机制类似于实验二中的 pp\_link,使用它和 env\_free\_list 来构造空闲进程链表。
  \item env\_id : 每个进程的 env\_id 都不一样,env\_id 是进程独一无二的标识符。
  \item env\_parent\_id : 该变量存储了创建本进程的进程 id。
  这样进程之间通过父子进程之间的关联可以形成一颗进程树。
  \item env\_status : 该变量只能在以下三个值中进行取值:
  \begin{itemize}
    \item ENV\_FREE : 表明该进程是不活动的,即该进程控制块处于进程空闲链表中。
    \item ENV\_NOT\_RUNNABLE : 表明该进程处于阻塞状态,
    处于该状态的进程往往在等待一定的条件才可以变为就绪状态从而被CPU调度。
    \item ENV\_RUNNABLE : 表明该进程处于就绪状态,正在等待被调度,
    但处于RUNNABLE状态的进程可以是正在运行的,也可能不在运行中。
  \end{itemize}
  \item env\_pgdir : 这个变量保存了该进程页目录的虚拟地址。
  \item env\_cr3 : 这个变量保存了该进程页目录的物理地址。
\end{itemize}

这里值得强调的一点就是\textbf{ENV\_RUNNABLE 状态并不代表进程一定在运行中,它也可能正处于调度队列中}。
而我们的进程调度也只会调度处于RUNNABLE状态的进程。

既然我们知道了进程控制块,我们就来认识一下进程控制块数组\textbf{envs}。在我们的实验中,
存放进程控制块的物理内存在系统启动后就要被分配好,并且这块内存不可被换出。
所以需要在系统启动后就要为envs数组分配内存,下面你就要完成这个重任

\begin{exercise}
  \begin{itemize}
    \item 修改pmap.c/mips\_vm\_init函数来为envs数组分配空间。
    \item envs数组包含 NENV 个Env结构体成员,你可以参考pmap.c中已经写过的\textbf{pages}数组空间的分配方式。
    \item 除了要为数组 envs 分配空间外,你还需要使用 pmap.c 中你填写过的一个内核态函数为其进行段映射,
    envs 数组应该被\textbf{UENVS}区域映射,你可以参考\textbf{./include/mmu.h}。	
  \end{itemize}
\end{exercise}

当然我们光有了存储进程控制块信息的envs还不够,我们还需要像lab2一样将空闲的env控制块按照链表形式“串”起来,便于后续分配ENV结构体对象,形成env\_free\_list。一开始我们的所有进程控制块都是空闲的,所以我们要把它们都“串”到env\_free\_list上去。

\begin{exercise}
仔细阅读注释,填写\textbf{env\_init}函数,注意要按照逆序插入。
\end{exercise}

在填写完env\_init函数后,我们对于envs的操作暂时就告一段落了,不过我们还有一个小问题没解决,注释里说明了要逆序插入,但是为什么呢?我们需要你来仔细思考,注释中已经给出了提示。

\begin{thinking}\label{think-env_init}
为什么我们在构造空闲进程链表时使用了逆序插入的方式?
\end{thinking}

\subsection{设置进程控制块}

做完上面那个小练习后,那么我们就可以开始利用\textbf{空闲进程链表env\_free\_list}创建进程来玩了。下面我们就来具体讲讲你应该如何创建一个进程\footnote{这里我们创建进程是指系统创建进程,不是指fork等进程“生”进程。我们将在lab4中接触另一种进程创建的方式。}。

进程创建的流程如下:

\begin{description}
  \item[第一步] 申请一个空闲的PCB,从env\_free\_list中索取一个空闲PCB块,这时候的PCB就像张白纸一样。
  \item[第二步] “纯手工打造”打造一个进程。在这种创建方式下,由于没有模板进程,所以进程拥有的所有信息都是手工设置的。
  而进程的信息又都存放于进程控制块中,所以我们需要手工初始化进程控制块。
  \item[第三步] 进程光有PCB的信息还没法跑起来,每个进程都有独立的地址空间。\label{process-3}
  所以,我们要为新进程分配资源,为新进程的程序和数据以及用户栈分配必要的内存空间。
  \item[第四步] 此时PCB已经被涂涂画画了很多东西,不再是一张白纸,把它从空闲链表里除名,就可以投入使用了。
\end{description}

当然,第二步的信息设置是本次实验的关键,那么下面让我们来结合注释看看这段代码

\begin{codeBoxWithCaption}{进程创建\label{code:env_alloc.c}}
  \inputminted[linenos]{c}{codes/env_alloc.c}
\end{codeBoxWithCaption}

相信你一眼看到第三条注释的时候一定会吐槽:“什么鬼,什么叫合适的值啊”。淡定,先别着急吐槽,先花半分钟的时间看一下第二条注释。

下面这个函数就是你在第二步中要使用的函数。在用之前,我们希望你仔细看注释,并对这个函数进行一些思考。

\begin{codeBoxWithCaption}{地址空间初始化\label{code:env_setup_vm.c}}
  \inputminted[linenos]{c}{codes/env_setup_vm.c}
\end{codeBoxWithCaption}

请在仔细看完上述函数的注释和其中的Hint后,来思考一下下面这些为你准备的问题

\begin{thinking}\label{think-env_setup_vm}
思考env\_ setup\_ vm函数:
\begin{itemize}
\item 第三点注释中的问题:为什么我们要执行\mintinline{c}|pgdir[i] = boot_pgdir[i]|这个赋值操作?换种说法,我们为什么要使用\mintinline{c}|boot_pgdir|作为一部分模板?(提示:mips虚拟空间布局)
\item {\small UTOP}和{\small ULIM}的含义分别是什么,在{\small UTOP}到{\small ULIM}的区域与其他用户区相比有什么最大的区别?
\item (选做)我们为什么要让\mintinline{c}|pgdir[PDX(UVPT)]=env_cr3?|(提示:结合系统自映射机制)
\end{itemize}
\end{thinking}

其实第三点注释问题本身要思考起来是有比较大跨度的,所以我们直接在这里给出笔者在反复思考与理解后所获得的心得,
鼓励你在写文档时可以进行更多思考与更深层的理解:)。

在我们的实验中,对于不同的进程而言,虚拟地址ULIM以上的地方,虚拟地址到物理地址的映射关系都是一样的。
因为这2G虚拟空间,不是由哪个进程管理的,而是由内核管理的!如果你仔细思索这句话,
可能会产生疑惑:“那为什么不能该内核管的时候让内核进程去管,该普通进程管的时候给普通进程去管,
非要混在一个地址空间里搞来搞去的呢?”这种想法是很好的,但是问题来了,在我们这种布局模式下,
有严格意义上的内核进程吗?

答案当然是否定的,这里我们要再讲清楚几个概念,方可解决这个问题。

首先来科普下虚拟空间的分配模式。我们知道,每一个进程都有4G的逻辑地址可以访问,
我们所熟知的系统不管是Linux还是Windows系统,都可以支持3G/1G模式或者2G/2G模式。
3G/1G模式即满32位的进程地址空间中,用户态占3G,内核态占1G。这些情况在进入内核态的时候叫做陷入内核,
因为\textbf{即使进入了内核态,还处在同一个地址空间中,并不切换CR3寄存器。}
但是!还有一种模式是4G/4G模式,内核单独占有一个4G的地址空间,所有的用户进程独享自己的4G地址空间,
这种模式下,在进入内核态的时候,叫做切换到内核,\textbf{因为需要切换CR3寄存器},
所以进入了\textbf{不同的地址空间}!

\begin{note}
用户态和内核态的概念相信大家也不陌生,内核态即计组实验中所提到的特权态,用户态就是非特权态。
mips汇编中使用一些特权指令如\mintinline{gas}|mtc0|、\mintinline{gas}|mfc0|、\mintinline{gas}|syscall|等都会陷入特权态(内核态)。
\end{note}

而我们这次实验,根据\mintinline{c}|./include/mmu.h|里面的布局来说,我们是2G/2G模式,
用户态占用2G,内核态占用2G。接下来,也是最容易产生混淆的地方,进程从用户态提升到内核态的过程,
操作系统中发生了什么?

是从当前进程切换成一个所谓的“内核进程”来管理一切了吗?不!还是一样的配方,还是一样的进程!
改变的其实只是进程的权限!我们打一个比方,你所在的班级里没有班长,
任何人都可以以合理的理由到老师那申请当临时班长。你是班里的成员吗?当然是的。
某天你申请当临时班长,申请成功了,那么现在你还是班里的成员吗?当然还是的。
那么你前后有什么区别呢?是权限的变化。可能你之前和其他成员是完全平等,互不干涉的。
那么现在你可以根据花名册点名,你可以安排班里的成员做些事情,你可以开班长会议等等。
那么我们能说你是班长吗?不能,因为你并不是永久的班长。但能说你拥有成为班长的资格吗?
当然可以,这种\textbf{成为临时班长的资格},我们可以粗略地认为它就是内核态的精髓所在。

而在操作系统中,每个完整的进程都拥有这种成为临时内核的资格,即所有的进程都可以发出申请变成内核态下运行的进程。
内核态下进程可访问的资源更多,更加自由。在之后我们会提到一种“申请”的方式,就叫做“系统调用”。

那么你现在应该能够理解为什么我们要将内核才能使用的虚页表为每个进程都拷贝一份了,在2G/2G这种布局模式下,
每个进程都会有\textbf{2G内核态}的虚拟地址,以便临时变身为“内核”行使大权。但是,在变身器使用之前,就算是奥特曼,
一样也只能访问自己的那2G用户态的虚拟地址。

那么这种微妙的关系应该类似于下面这种:(图中灰色代表不可用,白色代表可用)
\begin{figure}[htbp]
  \centering
  \includegraphics[width=14cm]{Process.png}
  \caption{进程页表与内核页表的关系}\label{fig:Process.png} 
\end{figure}


在上述的思考完成后,那么我们就可以直接在\textbf{env\_alloc} 第二步直接使用该函数了。
现在来解决一下刚才的问题,第三点所说的合适的值是什么?我们要设定哪些变量的值呢?

我们要设定的变量其实在\textbf{env\_alloc} 函数的提示中已经说的很清楚了,至于其合适的值,
相信仔细的你从函数的前面长长的注释里一定能获得足够的信息。当然我要讲的重点不在这里,
重点在我们已经给出的这个设置\mintinline{c}|e->env_tf.cp0_status = 0x10001004;|

这个设置很重要,很重要,很重要,重要到我们必须直接在代码中给出。为什么说它重要,各位看官且听我娓娓道来。

\begin{figure}[htbp]
  \centering
  \includegraphics[width=15cm]{3-R3000_SR.png}
  \caption{R3000的SR寄存器示意图}\label{fig:3-R3000_SR.png} 
\end{figure}

图\ref{fig:3-R3000_SR.png}是我们MIPSR3000里的SR(status register)寄存器示意图,就是我们在env\_tf里的cp0\_ status。

第28bit设置为1,表示处于用户模式下。

第12bit设置为1,表示4号中断可以被响应。

这些都是小case,下面讲的才是重点!

R3000的SR寄存器的低六位是一个二重栈的结构。
KUo和IEo是一组,每当中断发生的时候,硬件自动会将KUp和IEp的数值拷贝到这里;
KUp和IEp是一组,当中断发生的时候,硬件会把KUc和IEc的数值拷贝到这里。

其中KU表示是否位于内核模式下,为1表示位于内核模式下;
IE表示中断是否开启,为1表示开启,否则不开启\footnote{我们的实验是不支持中断嵌套的,
所以内核态时是不可以开启中断的。}。

而每当rfe指令调用的时候,就会进行上面操作的\textbf{逆操作}。
我们现在先不管为何,但是已经知道,下面这一段代码在\textbf{运行第一个进程前是一定要执行的},
\label{env_pop_tf}所以就一定会执行\mintinline{gas}|rfe|这条指令。

\begin{minted}[linenos]{gas}
lw   k0,TF_STATUS(k0)        #恢复CP0_STATUS寄存器    
mtc0 k0,CP0_STATUS
j    k1
rfe
\end{minted}

现在你可能就懂了为何我们status后六位是设置为\mintinline{c}|000100b|了。
当运行第一个进程前,运行上述代码到\mintinline{gas}|rfe|的时候,就会将KUp和IEp拷贝回KUc和IEc,令status为 
\mintinline{c}|000001b|,最后两位KUc,IEc为\textbf{[0,1]},表示开启了中断。之后第一个进程成功运行,
这时操作系统也可以正常响应中断,Nice!

\begin{note}
{\small 关于MIPSR3000版本SR寄存器功能的英文原文描述:
The status register is one of the more important registers. The register has several
fields. The current Kernel/User (KUc) flag states whether the CPU is in kernel mode.
The current Interrupt Enable (IEc) flag states whether external interrupts are turned on.
If cleared then external interrupts are ignored until the flag is set again. In an exception
these flags are copied to previous Kernel/User and Interrupt Enable (KUp and IEp) and
then cleared so the system moves to a kernel mode with external interrupts disabled.
The Return From Exception instruction writes the previous flags to the current flags.}
\end{note}

当然我们从注释也能看出,第四步除了需要设置\mintinline{c}|cp0status|以外,还需要设置栈指针。
在MIPS中,栈寄存器是第29号寄存器,注意这里的栈是用户栈,不是内核栈。

\begin{exercise}
根据上面的提示与代码注释,填写 \textbf{env\_ alloc} 函数。
\end{exercise}

%好,填到这里你可能有些疲倦了,我们来讲个笑话。

%\begin{note}
%从前有一个年轻人,他小时候就表示他渴望将来能成为一个伟大的作家。当被问及怎样才算是“伟大”时,
%他说:“我要写出全世界的人都会阅读的作品,这些作品会唤起人们内心最深处的情感,
%这些作品会让他们在痛苦和愤怒之中尖叫、哭泣和咆哮!”

%此人目前就职于微软公司,负责撰写错误信息。
%\end{note}

\subsection{加载二进制镜像}
%开心一笑过后,
我们继续来完成我们的实验。下面这个函数还是蛮难填的呢,所以大家一定要跟紧我的步伐,我们慢慢来分析一下这个函数。

我们在\hyperref[process-3]{进程创建}第三点曾提到过,我们需要为\textbf{新进程的程序}分配空间来容纳程序代码。
那么下面我需要有两个函数来一起完成这个任务

\begin{codeBoxWithCaption}{加载镜像映射\label{code:load_icode_mapper.c}}
  \inputminted[linenos]{c}{codes/load_icode_mapper.c}
\end{codeBoxWithCaption}

为了完成这个函数,我们接下来再补充一点关于ELF的知识。

前面在讲解内核加载的时候,我们曾简要说明过ELF的加载过程。这里,我们再做一些补充说明。要想正确加载一个ELF文件到内存,
只需将ELF文件中所有需要加载的segment加载到对应的虚地址上即可。为了降低难度,我们已经写好了用于解析ELF文件的代码
(lib/kernel\_elfloader.c),你可以直接调用相应函数获取ELF文件的各项信息,并完成加载过程。该函数的原型如下:

\begin{minted}[linenos]{c}
// binary为整个待加载的ELF文件。size为该ELF文件的大小。
// entry_point是一个u_long变量的地址(相当于引用),解析出的入口地址会被存入到该位置
int load_elf(u_char *binary, int size, u_long *entry_point, void *user_data,
             int (*map)(u_long va, u_int32_t sgsize,
                        u_char *bin, u_int32_t bin_size, void *user_data))
\end{minted}

我们来着重解释一下load\_elf()函数的设计,以及最后两个参数的作用。为了让你有机会完成加载可执行文件到内存的过程,
load\_elf()函数只完成了解析ELF文件的部分,而把将ELF文件的各个segment加载到内存的工作留给了你。
为了达到这一目标,load\_elf()的最后两个参数用于接受一个你的自定义函数以及你想传递给你的自定义函数的额外参数。
每当load\_elf()函数解析到一个需要加载的segment,会将ELF文件里与加载有关的信息作为参数传递给你的自定义函数。
你的自定义函数完成加载单个segment的过程。

为了进一步简化你的理解难度,我们已经为你定义好了这个“自定义函数”的框架。如代码\ref{code:load_icode_mapper.c}所示。
load\_elf()函数会从ELF文件文件中解析出每个segment的四个信息:va(该段需要被加载到的虚地址)、sgsize(该段在内存中的大小)、
bin(该段在ELF文件中的内容)、bin\_size(该段在文件中的大小),并将这些信息传给我们的“自定义函数”。

接下来,你只需要完成以下两个步骤:

\begin{description}
  \item[第一步] 加载该段在ELF文件中的所有内容到内存。
  \item[第二步] 如果该段在文件中的内容的大小达不到该段在内存中所应有的大小,那么余下的部分用0来填充。
\end{description}

也许机灵的你发现了一个很无语的情况:我们并没有真正解释清楚user\_data这个参数的作用。最后一个参数是一个函数指针,
用于将我们的自定义函数传入进去。但这个诡异的user\_data到底是做什么用的呢?这样设计又是为了什么呢?
很不幸,这个问题我们决定留给你来思考。

\begin{thinking}\label{think-user-data}
思考user\_data这个参数的作用。没有这个参数可不可以?为什么?(如果你能说明哪些应用场景中可能会应用这种设计就更好了。
可以举一个实际的库中的例子)
\end{thinking}

思考完这一点,我们就可以进入这一小节的练习部分了。

\begin{exercise}
通过上面补充的知识与注释,填充 \textbf{load\_ icode\_ mapper} 函数。
\end{exercise}

现在我们已经完成了补充部分最难的一个函数,那么下面我们完成这个函数后,
就能真正实现把二进制镜像加载进内存的任务了。

\begin{codeBoxWithCaption}{完整加载镜像\label{code:load_icode.c}}
  \inputminted[linenos]{c}{codes/load_icode.c}
\end{codeBoxWithCaption}

现在我们来根据注释一步一步完成这个函数。
在第二步我们要用第一步申请的页面来初始化一个进程的栈,根据注释你应当可以轻松完成。
这里我们只讲第三步的注释所代表的内容,其余你可以根据注释中的提示来完成。

第三步通过调用我们预先为你准备好的load\_elf()函数来将ELF文件真正加载到内存中。
这里仅做一点提醒:请将load\_icode\_mapper()这个函数作为参数传入到load\_elf()中。
其余的参数在前面已经解释过了,就不再赘述了。

\begin{exercise}
通过补充的ELF知识与注释,填充 \textbf{load\_ icode} 函数。
\end{exercise}

这里的\mintinline{c}|e->env_tf.pc|是什么呢?就是在我们计组中反复强调的甚为重要的\mintinline{c}|PC|。
它指示着进程当前指令所处的位置。你应该知道,冯诺依曼体系结构的一大特点就是
:程序预存储,计算机自动执行。我们要运行的进程的代码段预先被载入到了\textbf{entry\_ point}为起点的
内存中,当我们运行进程时,CPU将自动从pc所指的位置开始执行二进制码。

\begin{thinking}\label{think-位置}
思考上面这一段话,并根据自己在\textbf{lab2}中的理解,回答:
  \begin{itemize}
  \item 我们这里出现的"指令位置"的概念,你认为该概念是针对虚拟空间,还是物理内存所定义的呢?
  \item 你觉得\mintinline{c}|entry_point|其值对于每个进程是否一样?该如何理解这种统一或不同?
  \item 从布局图中找到你认为最有可能是\mintinline{c}|entry_point|的值。
  \end{itemize}
\end{thinking}

思考完这一点后,下面我们来准备准备可以真正创建进程了。

\subsection{创建进程}

创建进程的过程很简单,就是实现对上述个别函数的封装,\textbf{分配一个新的Env结构体,设置进程控制块,并将二进制代码载入到对应地址空间}即可完成。好好思考上面的函数,我们需要用到哪些函数来做这几件事?

\begin{exercise}
根据提示,完成 \textbf{env\_create} 函数的填写。
\end{exercise}

当然提到创建进程,我们还需要提到一个封装好的宏命令

\begin{minted}[linenos]{c}
#define ENV_CREATE(x) \
{ \
    extern u_char binary_##x##_start[];\
    extern u_int binary_##x##_size; \
    env_create(binary_##x##_start, \
        (u_int)binary_##x##_size); \
}
\end{minted}

这个宏里的语法大家可能以前没有见过,这里解释一下\mintinline{c}|##|代表拼接,例如\footnote{这个例子是转载的,出处为\url{http://www.cnblogs.com/hnrainll/archive/2012/08/15/2640558.html},想深入了解的同学可以参考这篇博客}

\begin{minted}[linenos]{c}
#define CONS(a,b) int(a##e##b) 
int main() 
{
    printf("%d\n", CONS(2,3));  // 2e3 输出:2000 
    return 0; 
}
\end{minted}

好,那么现在我们就得手工在我们的\mintinline{c}|init/init.c|里面加两句话来初始化创建两个进程

\begin{minted}[linenos]{c}
ENV_CREATE(user_A);
ENV_CREATE(user_B);
\end{minted}

这两句话加在哪里呢?那就需要你翻代码来寻找啦~

\begin{exercise}
根据注释与理解,将上述两条进程创建命令加入 \textbf{init/init.c} 中。
\end{exercise}

做完这些,是不是迫不及待地想要跑个进程看看能否成功?等我们完成下面这个函数,就可以开始第一部分的自我测试了!

\subsection{进程运行与切换}

\begin{codeBoxWithCaption}{进程的运行\label{code:env_run.c}}
  \inputminted[linenos]{c}{codes/env_run.c}
\end{codeBoxWithCaption}

刚刚说到的load\_icode 是为数不多的坑函数之一,env\_run 也是,而且其实按程度来讲可能更甚一筹。

env\_run,是进程运行使用的基本函数,它包括两部分:
\begin{itemize}
  \item 一部分是保存当前进程上下文(\textbf{如果当前没有运行的进程就跳过这一步})
  \item 另一部分就是恢复要启动的进程的上下文,然后运行该进程。
\end{itemize}

\begin{note}
进程上下文说来就是一个环境,相对于进程而言,就是进程执行时的环境。具体来说就是各个变量和数据,包括所有的寄存器变量、内存信息等。
\end{note}

其实我们这里运行一个新进程往往意味着是进程切换,而不是单纯的进程运行。进程切换,
人如其名,就是当前进程停下工作,让出CPU处理器来运行另外的进程。
那么要理解进程切换,我们就要知道进程切换时系统需要做些什么。Alt+Tab可以吗?当然不可以。
实际上进程切换的时候,为了保证下一次进入这个进程的时候我们不会再“从头来过”,
而是有记忆地从离开的地方继续往后走,我们要保存一些信息,那么,
需要保存什么信息呢?理所当然地想想,你可能会想到下面两种需要保存的状态:
\begin{description}
\item[进程本身的状态]
\item[进程周围的环境状态]
\end{description}

那么我们先解决一个问题,进程本身的状态需要记录吗?
进程本身的状态无非就是进程块里面那几个东西,包括

\textbf{env\_id,env\_parent\_id,env\_pgdir,env\_cr3...}

或许你会有所疑问,\textbf{env\_tf}不算是进程本身的状态吗?按笔者的理解来说,
是不算的。env\_tf保存的是进程上下文,相当于我们的第二点,进程周围的环境状态。

我们仔细思索一下,就能发现,进程本身的状态在进程切换的时候是不会变化的。
(我们不会去毁灭一个进程块,进程块跟我们又没仇。)
会变的也是需要我们保存的实际上是进程的环境信息。

这样或许你就能明白run代码中的第一句注释的含义了:/*Step 1: save register state of curenv. */


那么你可能会想,进程运行到某个时刻,它的上下文——所谓的CPU的寄存器在哪呢?我们又该如何保存?
在lab3中,我们在本实验里的寄存器状态保存的地方是TIMESTACK区域。
\mint{c}|struct Trapframe  *old;|
\mint{c}|old = (struct Trapframe *)(TIMESTACK - sizeof(struct Trapframe));|
这个old就是当前进程的上下文所存放的区域。那么第一步注释还说到,让我们参考\mintinline{c}|env_destroy|
,其实就是让我们把old区域的东西\textbf{拷贝到当前进程的env\_ tf中},以达到保存进程上下文的效果。
那么我们还有一点很关键,就是当我们返回到该进程执行的时候,应该从哪条指令开始执行?
即当前进程上下文中的pc应该设为什么值?这将留给聪明的你去思考。

\begin{thinking}\label{think-pc}
思考一下,要保存的进程上下文中的\mintinline{c}|env_tf.pc|的值应该设置为多少?为什么要这样设置?
\end{thinking}

思考完上面的,我们沿着注释再一路向下,后面好像没有什么很难的地方了。根据提示也完全能够做出来。
但是我们还有一点坑没填,我们忽略了 \textbf{env\_pop\_tf}函数。

env\_pop\_tf 是定义在 lib/env\_asm.S 中的一个汇编函数。这个函数也可以用来解释:为什么启动第一个进程前一定会执行
\mintinline{gas}|rfe|汇编指令。但是我们本次思考的重点不在这里,重点在于TIMESTACK。
请仔细地看看这个函数,你或许能看出什么关于TIMESTACK的端倪。
TIMESTACK问题可能将是你在本实验中需要思考时间最久的问题,希望你能和小伙伴积极交流,努力寻找实验
代码来支撑你的看法与观点,鼓励提出新奇的想法!

\begin{thinking}\label{think-TIMESTACK}
思考TIMESTACK的含义,并找出相关语句与证明来回答以下关于TIMESTACK的问题:
\begin{itemize}
  \item 请给出一个你认为合适的TIMESTACK的定义
  \item 请为你的定义在实验中找出合适的代码段作为证据(请对代码段进行分析)
  \item 思考TIMESTACK和第18行的KERNEL\_SP 的含义有何不同
\end{itemize}
\end{thinking}

\begin{exercise}
根据补充说明,填充完成 \textbf{env\_run} 函数。
\end{exercise}

至此,我们第一部分的工作已经完成了!第二部分的代码量很少,但是不可或缺!休息一下,我们继续奋斗!

\section{中断与异常}

之前我们在学习计组的时候已经学习了异常和中断的概念,所以这里我们不再将概念作为主要介绍内容。
\begin{note}
我们实验里认为凡是引起控制流突变的都叫做异常,而中断仅仅是异常的一种,并且是仅有的一种异步异常。
\end{note}

我们可以通过一个简单的图来认识一下异常的产生与返回(见图\ref{fig:3-exception.png})。
\begin{figure}[htbp]
  \centering
  \includegraphics[width=15cm]{3-exception.png}
  \caption{异常处理图示}\label{fig:3-exception.png} 
\end{figure}

\subsection{异常的分发}

每当发生异常的时候,一般来说,处理器会进入一个用于分发异常的程序,
这个程序的作用就是检测发生了哪种异常,并调用相应的异常处理程序。
一般来说,这个程序会被要求放在固定的某个物理地址上(根据处理器的区别有所不同),
以保证处理器能在检测到异常时正确地跳转到那里。这个分发程序可以认为是操作系统的一部分。

代码\ref{code:exec_vec3}就是异常分发代码,
我们先将下面代码填充到我们的\mintinline{c}|start.S|的开头,然后我们来分析一下。

\label{code:exec_vec3}
\begin{minted}[linenos]{gas}
.section .text.exc_vec3
NESTED(except_vec3, 0, sp)
     .set noat
     .set noreorder
  1:
     mfc0 k1,CP0_CAUSE
     la   k0,exception_handlers
     andi k1,0x7c
     addu k0,k1
     lw   k0,(k0)
     NOP
     jr   k0
     nop
END(except_vec3)
     .set at
\end{minted}

\begin{exercise}
将异常分发代码填入 \textbf{init/start.S} 合适的部分。
\end{exercise}

这段异常分发代码的作用流程简述如下:
\begin{enumerate}
  \item 取得异常码,这是区别不同异常的重要标志。
  \item 以得到的异常码作为索引去exception\_handlers数组中找到对应的中断处理函数,后文中会有涉及。
  \item 跳转到对应的中断处理函数中,从而响应了异常,并将异常交给了对应的异常处理函数去处理
\end{enumerate}

\begin{figure}[htbp]
  \centering
  \includegraphics[width=15cm]{3-CauseRegister.png}
  \caption{CR寄存器}\label{fig:3-CauseRegister.png} 
\end{figure}
图\ref{fig:3-CauseRegister.png}是MIPS3000中Cause Register寄存器。
其中保存着CPU中哪一些中断或者异常已经发生。bit2~bit6保存着异常码,也就是根据异常码来识别具体哪一个异常发生了。
bit8~bit15保存着哪一些中断发生了。其他的一些位含义在此不涉及,可参看MIPS开发手册。

这个.text.exec\_vec3 段将通过链接器放到特定的位置,在R3000中要求是放到0x80000080地址处,
这个地址处存放的是异常处理程序的入口地址。一旦CPU发生异常,就会自动跳转到0x80000080地址处,
开始执行,下面我们将.text.exec\_vec3 放到该位置,一旦异常发生,就会引起该段代码的执行,
而该段代码就是一个分发处理异常的过程。

所以我们要在我们的lds中增加这么一段,从而可以让R3000出现异常时自动跳转到异常分发代码处。
\begin{minted}[linenos]{c}
. = 0x80000080;
.except_vec3 : {
    *(.text.exc_vec3)
}
\end{minted}


\begin{exercise}
将lds代码补全使得异常后可以跳到异常分发代码。
\end{exercise}


\subsection{异常向量组}
我们刚才提到了异常的分发,要寻找到exception\_handlers 数组中的中断处理函数,
而exception\_handlers 就是所谓的异常向量组。

下面我们跟随\mintinline{c}|trap_init(lib/traps.c)|函数来看一下,异常向量组里存放的是些什么?

\begin{minted}[linenos]{c}
extern void handle_int();
extern void handle_reserved();
extern void handle_tlb();
extern void handle_sys();
extern void handle_mod();
unsigned long exception_handlers[32];

void trap_init()
{
    int i;

    for (i = 0; i < 32; i++) {
        set_except_vector(i, handle_reserved);
    }

    set_except_vector(0, handle_int);
    set_except_vector(1, handle_mod);
    set_except_vector(2, handle_tlb);
    set_except_vector(3, handle_tlb);
    set_except_vector(8, handle_sys);
}
void *set_except_vector(int n, void *addr)
{
    unsigned long handler = (unsigned long)addr;
    unsigned long old_handler = exception_handlers[n];
    exception_handlers[n] = handler;
    return (void *)old_handler;
}

\end{minted}

实际上呢,这个函数实现了对全局变量exception\_handlers[32]数组初始化的工作,
其实就是把相应的处理函数的地址填到对应数组项中。
主要初始化
\begin{description}
  \item[0号异常]的处理函数为handle\_int,
  \item[1号异常]的处理函数为handle\_mod,
  \item[2号异常]的处理函数为handle\_tlb,
  \item[3号异常]的处理函数为handle\_tlb,
  \item[8号异常]的处理函数为handle\_sys,
\end{description}
一旦初始化结束,有异常产生,那么其对应的处理函数就会得到执行。而在我们的实验中,我们主要是
要使用0号异常,即中断异常的处理函数。因为我们接下来要做的,就是要产生时钟中断。

\subsection{时钟中断}

希望你还没有忘记在计组实验中所接触到的\textbf{定时器}这个概念。或许你当时对定时器的作用会有疑惑,
为了防止你继续迷糊不清,我们下面来简单介绍一下时钟中断的概念。

时钟中断和操作系统的时间片轮转算法是紧密相关的。时间片轮转调度是一种很公平的算法。
每个进程被分配一个时间段,称作它的时间片,即该进程允许运行的时间。如果在时间片结束时进程还在运行,
则该进程将挂起,切换到另一个进程运行。那么CPU是如何知晓一个进程的时间片结束的呢?就是通过定时器产生的时钟中断。
当时钟中断产生时,当前运行的进程被挂起,我们需要在调度队列中选取一个合适的进程运行。
如何“选取”,就要涉及到进程的调度了。

要产生时钟中断,我们不仅要了解中断的产生与处理。我们还要了解gxemul是如何模拟时钟中断的。
初始化时钟主要是在 kclock\_init 函数中,该函数主要调用set\_timer 函数,完成如下操作:
\begin{itemize}
  \item 首先向0xb5000100位置写入1,其中0xb5000000是模拟器(gxemul)映射实时钟的位置。偏移量为0x100表示来设置实时钟中断的频率,1则表示1秒钟中断1次,如果写入0,表示关闭实时钟。实时钟对于R3000来说绑定到了4号中断上,故这段代码其实主要用来触发了4号中断。注意这里的中断号和异常号是不一样的概念,我们实验的异常包括中断。
  \item 一旦实时钟中断产生,就会触发MIPS中断,从而MIPS将PC指向\mintinline{c}|0x80000080|,从而跳转到\mintinline{c}|.text.exc_vec3|代码段执行。对于实时钟引起的中断,通过text.exc\_vec3代码段的分发,最终会调用handle\_ int函数来处理实时钟中断。
  \item 在handle\_ int判断\mintinline{c}|CP0_CAUSE|寄存器是不是对应的4号中断位引发的中断,如果是,则执行中断服务函数timer\_ irq。
  \item 在timer\_ irq里直接跳转到sched\_ yield中执行。而这个函数就是我们将要补充的调度函数。
\end{itemize}

以上就是我们时钟中断的产生与中断处理流程,我们在这里要完成下面的任务以顺利产生时钟中断。

\begin{exercise}
通过上面的描述,补充 \textbf{ kclock\_init } 函数。
\end{exercise}

\subsection{进程调度}

通过上面的描述,我们知道了,其实在时钟中断产生时,最终是调用了sched\_ yield函数来进行进程的调度,
这个函数在\mintinline{c}|lib/sched.c|中所定义。这个函数就是我们本次最后要写的调度函数。
调度的算法很简单,就是时间片轮转的算法,没有优先级,根据注释就可以轻松写出。

\begin{exercise}
根据注释,完成 \textbf{sched\_yield }函数的补充。
\end{exercise}

至此,我们的实验三就算是圆满完成了。

\section{实验正确结果}
如果你按流程做下来并且做的结果正确的话,你运行之后应该会出现这样的结果

\begin{minted}[linenos]{c}
init.c: mips_init() is called

Physical memory: 65536K available, base = 65536K, extended = 0K

to memory 80401000 for struct page directory.

to memory 80431000 for struct Pages.

mips_vm_init:boot_pgdir is 80400000

pmap.c:  mips vm init success

panic at init.c:27: ^^^^^^^^^^^^^^^^^^^^^^^^^^^^^^^^^^^^^

2 2 2 2 2 2 2 2 2 2 2 2 2 2 2 2 2 2 2 2 2 2 2 2 2 2 2 2 2 
1 1 1 1 1 1 1 1 1 1 1 1 1 1 1 1 1 1 1 1 1 1 1 1 1 1 1 1   
2 2 2 2 2 2 2 2 2 2 2 2 2 2 2 2 2 2 2 2 2 2 2 2 2 2 2 
1 1 1 1 1 1 1 1 1 1 1 1 1 1 1 1 1 1 1 1 1 1 1 1 1 1 1 1 1 1 
2 2 2 2 2 2 2 2 2 2 2 2 2 2 2 2 2 2 2 2 2 2 2 2 2 2 2 2 2 
1 1 1 1 1 1 1 1 1 1 1 1 1 1 1 1 1 1 1 1 1 1 1 1 1 1 1 1 1 
\end{minted}

当然不会这么整齐,且没有换行,只是交替输出2和1而已~当然还不能放过你,你还需要再思考一部分内容

\begin{thinking}\label{think-调度}
思考一下你的调度程序,这种调度方式由于某种不可避免的缺陷而造成对进程的不公平。
  \begin{itemize}
    \item 这种不公平是如何产生的?
    \item 如果实验确定只运行两个进程,你如何改进可以降低这种不公平?
  \end{itemize}
\end{thinking}

\section{实验思考}

\begin{itemize}
\item \hyperref[think-env_init]{\textbf{\textcolor{baseB}{思考-init的逆序插入}}}
\item \hyperref[think-env_setup_vm]{\textbf{\textcolor{baseB}{思考-地址空间初始化}}}
\item \hyperref[think-user-data]{\textbf{\textcolor{baseB}{思考-user\_data的作用}}}
\item \hyperref[think-位置]{\textbf{\textcolor{baseB}{思考-位置的含义}}}
\item \hyperref[think-pc]{\textbf{\textcolor{baseB}{思考-进程上下文的PC值}}}
\item \hyperref[think-TIMESTACK]{\textbf{\textcolor{baseB}{思考-TIMESTACK的含义}}}
\item \hyperref[think-调度]{\textbf{\textcolor{baseB}{思考-不公的调度}}}
\end{itemize}

\section{挑战性任务}
我们实验中使用的调度算法是很简单的,那么现在希望你可以向内核中添加自己的调度策略,比如带优先级的时间片轮转算法。
修改调度算法,并利用测试的现象说明调度算法是正确的。如果你做了这部分的挑战性任务,请将代码提交到分支
\textbf{lab3-challenge}并推送到服务器上,并在实验文档中写入新的调度策略对于结果的影响并说明为什么,
可以自定义测试程序。
\chapter{系统调用与fork}

\section{实验目的}
  \begin{enumerate}
    \item 掌握系统调用的概念及流程
    \item 实现进程间通讯机制
    \item 实现fork函数
  \end{enumerate}

在系统中真正被所有进程都使用的内核通信方式是系统调用,例如当进程请求内核服务时,就使用的是系统调用。
一般情况下,进程是不能够存取系统内核的,它不能存取内核使用的内存段,也不能调用内核函数,
CPU的硬件结构保证了这一点,只有系统调用是一个例外。
进程使用寄存器中适当的值跳转到内核中事先定义好的代码中执行(当然,这些代码是只读的)。
在这一节的实验中,我们需要实现系统调用机制,并再此基础上实现进程间通信(ipc)机制和fork。

\section{系统调用(System Call)}
本节中,我们着重讨论系统调用的作用,并完成实现相关的内容。

\subsection{一探到底,系统调用的来龙去脉}
说起系统调用,你冒出的第一个问题一定是:系统调用到底长什么样子?为了一探究竟,我们选择一个即为简单的程序作为实验对象。
在这个程序中,我们通过puts来输出一个字符串到终端。

\begin{minted}[linenos]{c}
#include <stdio.h>

int main() {
        puts("Hello World!\n");
        return 0;
}
\end{minted}

\begin{note}
如果你还记得C语言课上关于标准输出的相关知识的话,你一定知道在C语言中,终端被抽象为了标准输出文件stdout。
通过向标准输出文件写东西,就可以输出内容到屏幕。而向文件写入内容是通过write系统调用完成的。
因此,我们选择通过观察puts函数,来探究系统调用的奥秘。
\end{note}

我们通过GDB进行单步调试,逐步深入到函数之中,观察puts具体的调用过程\footnote{这里为了方便大家在自己的机器上重现,
我们选用了Linux X86\_64平台作为实验平台}。运行GDB,将断点设置在puts这条语句上,并通过stepi指令\footnote{为了加快调试进程,
可以尝试stepi N指令,N的位置填写任意数字均可。这样每次会执行N条机器指令。笔者使用的是stepi 10。}
单步进入到函数中。当程序到达write函数时停下,因为write正是Linux的一条系统调用。我们打印出此时的函数调用栈,
可以看出,C标准库中的puts函数实际上通过了很多层函数调用,最终调用到了底层的write函数进行真正的屏幕打印操作。

\begin{minted}[linenos]{objdump}
(gdb) 
0x00007ffff7b1b4e0 in write () from /lib64/libc.so.6
(gdb) backtrace
#0  0x00007ffff7b1b4e0 in write () from /lib64/libc.so.6
#1  0x00007ffff7ab340f in _IO_file_write () from /lib64/libc.so.6
#2  0x00007ffff7ab2aa3 in ?? () from /lib64/libc.so.6
#3  0x00007ffff7ab4299 in _IO_do_write () from /lib64/libc.so.6
#4  0x00007ffff7ab462b in _IO_file_overflow () from /lib64/libc.so.6
#5  0x00007ffff7ab5361 in _IO_default_xsputn () from /lib64/libc.so.6
#6  0x00007ffff7ab3992 in _IO_file_xsputn () from /lib64/libc.so.6
#7  0x00007ffff7aaa4ef in puts () from /lib64/libc.so.6
#8  0x0000000000400564 in main () at test.c:4
\end{minted}

通过gdb显示的信息,我们可以看到,这个write()函数是在libc.so这个动态链接库中的,也就是说,它仍然是C库中的函数,
而不是内核中的函数。因此,该write()函数依旧是个用户空间函数。为了彻底揭开这个函数的秘密,我们对其进行反汇编。

\begin{minted}[linenos]{objdump}
(gdb) disassemble 0x00007ffff7b1b4e0
Dump of assembler code for function write:
=> 0x00007ffff7b1b4e0 <+0>:     cmpl   $0x0,0x2bf759(%rip)        # 0x7ffff7ddac40
   0x00007ffff7b1b4e7 <+7>:     jne    0x7ffff7b1b4f9 <write+25>
   0x00007ffff7b1b4e9 <+9>:     mov    $0x1,%eax
   0x00007ffff7b1b4ee <+14>:    syscall 
   0x00007ffff7b1b4f0 <+16>:    cmp    \$0xfffffffffffff001,%rax
   0x00007ffff7b1b4f6 <+22>:    jae    0x7ffff7b1b529 <write+73>
   0x00007ffff7b1b4f8 <+24>:    retq
End of assembler dump.
\end{minted}

通过gdb的反汇编功能,我们可以看到,这个函数最终执行了syscall这个极为特殊的指令。从它的名字我们就能够猜出它的用途,
它使得进程陷入到内核态中,执行内核中的相应函数,完成相应的功能。在系统调用返回后,用户空间的相关函数会将系统调用的结果,
通过一系列的过程,最终返回给用户程序。

由此我们可以看到,系统调用实际上是操作系统和用户空间的一组接口。用户空间的程序通过系统调用来访问操作系统的一些服务,
谋求操作系统提供必要的帮助。

在进行了上面的一系列探究后,我们将我们的发现罗列出来,整理一下我们的思路:
\begin{itemize}
  \item 存在一些只能由操作系统来完成的操作(如读写设备、创建进程等)。
  \item 用户程序要实现一些功能(比如执行另一个程序、读写文件),必须依赖操作系统的帮助。
  \item C标准库中的一些函数的实现必须依赖于操作系统(如我们所探究的puts函数)。
  \item 通过执行syscall指令,我们可以陷入到内核态,请求操作系统的一些服务。
  \item 直接使用操作系统的功能是很繁复的(每次都需要设置必要的寄存器,并执行syscall指令)
\end{itemize}

之后,我们再来整理一下调用C标注库中的puts函数的过程中发生了什么:
\begin{enumerate}
  \item 调用puts函数
  \item 在一系列的函数调用后,最终调用了write函数。
  \item write函数设置了为寄存器设置了相应的值,并执行了syscall指令。
  \item 进入内核态,内核中相应的函数或服务被执行。
  \item 回到用户态的write函数中,将系统调用的结果从相关的寄存器中取回,并返回。
  \item 再次经过一系列的返回过程后,回到了put函数中。
  \item puts函数返回。
\end{enumerate}

综合上面这些内容,相信你一定已经发现了其中的巧妙之处。操作系统将自己所能够提供的服务以系统调用的方式提供给用户空间。
用户程序即可通过操作系统来完成一些特殊的操作。同时,所有的特殊操作就全部在操作系统的掌控之中了
(因为用户程序只能通过由操作系统提供的系统调用来完成这些操作,所以操作系统可以确保用户不破坏系统的稳定)。
而直接使用这些系统调用较为麻烦,于是由产生了用户空间的一系列API,如POSIX、C标准库等,它们在系统调用的基础上,
实现更多更高级的常用功能,使得用户在编写程序时不用再处理这些繁琐而复杂的底层操作,
而是直接通过调用高层次的API就能实现各种功能。通过这样巧妙的层次划分,使得程序更为灵活,也具有了更好的可移植性。
对于用户程序来说,只要自己所依赖的API不变,无论底层的系统调用如何变化,都不会对自己造成影响,
使得程序更易于在不同的系统间移植。整个结构如图\ref{fig:api-and-syscall}所示。

\begin{figure}[htbp]
  \centering
  \includegraphics[width=7cm]{api-and-syscall.eps}
  \caption{API、系统调用层次结构}\label{fig:api-and-syscall} 
\end{figure}

\subsection{系统调用机制的实现}
在发现了系统调用的本质之后,我们就要着手在我们的小操作系统中实现一套系统调用机制了。不过,不要着急,为了使得后面的思路更清晰,
我们先来整理一下系统调用的流程:
\begin{enumerate}
  \item 调用库函数
  \item 调用一个用户空间的系统调用函数(用于封装设置寄存器、传参等繁复的操作)。
  \item 内核空间系统调用函数被执行。
  \item 从相应寄存器中收集返回值,并返回到库函数中。
  \item 库函数返回,回到用户程序
\end{enumerate}

也就是说,为了实现系统调用机制并实现我们自己的系统调用,我们需要找到内核空间中的系统调用函数,以及对应的用户空间系统调用函数。
不难发现,内核空间的系统调用实现在lib/syscall\_all.c中。在这段代码中,有大量的以sys开头的函数,这些函数便是内核空间的
系统调用的实现。

之后,我们再看用户空间的user/syscall\_lib.c,发现了一堆以syscall开头的函数,每个函数都根据传入的参数,调用了msyscall函数。
看到这里,你想到了什么?msyscall中一定是实际执行设置寄存器、执行syscall指令一类的底层操作的函数。
我们看到的这些syscall开头的函数为用户封装了这些繁复的过程。

这些syscall函数已经被预先实现好了,我们来看看msyscall函数。这个函数位于user/syscall\_wrap.S。
msyscall需要实现将必要的参数压栈,并执行syscall指令陷入内核态。

\begin{exercise}
填写msyscall,使得系统调用机制可以正常工作。
\end{exercise}

看到这个exercise你是不是觉得有些无语?完全没有可以参考的信息啊!我们从何知道这个msyscall如何填写呢?
仔细思考一下,msyscall的主要工作是设置栈、寄存器等,为后面的系统调用设置好参数。也就是说,想知道msyscall需要做些什么,
我们只需看一下后面内核中的相关代码是如何使用这些参数的就OK了。知道了内核用了什么,自然也就知道我们应当设置些什么。
内核处理系统调用的中断处理程序位于lib/syscall.S中,相应的注释已经预先加在了代码中,如代码\ref{code:handlesys.S}所示。

\begin{codeBoxWithCaption}{系统调用服务程序\label{code:handlesys.S}}
  \inputminted[linenos]{gas}{codes/handlesys.S}
\end{codeBoxWithCaption}

通过上面的代码,我们可以知道,所有的参数都应当按照顺序放入栈中。根据MIPS的ABI标准,前4个参数通过a0-a3寄存器传递,
后面的参数通过栈了传递。值得注意的是,标准要求程序必须分配足够容纳\textbf{所有}参数的栈空间。也就是说,尽管在函数调用的过程中,
前四个参数是通过寄存器来传递的,但程序依旧需要为其分配栈空间。也就是说,
GCC在生成调用msyscall函数的代码时已经按照标准为我们开了一个足够大的栈,我们在msyscall中只需将前四个参数压入到栈中即可,
无需再自行分配栈空间。

完成msyscall函数后,我们的小内核的系统调用机制便可以正常工作了。

\subsection{基础系统调用函数}

在系统调用机制搞定之后,我们自然是要弄几个系统调用玩一玩了。我们实现些什么系统调用呢?打开 lib/syscall\_all.c,可以看到玲琅满目的系统调用函数等着我们去填写。这些系统调用都是基础的系统调用,不论是之后的IPC还是fork,都需要这些基础的系统调用作为支撑。

首先我们看向sys\_mem\_alloc。

[Ping... Unreachabl]

\section{进程间通信机制(Inter-Process Communication)}
在系统调用机制搞定之后,我们自然是要弄几个系统调用玩一玩了。作为一个微内核系统,我们要来实现个什么系统调用呢?
没错,当然是IPC了。IPC可是微内核最重要的机制之一了。

\begin{note}
上世纪末,微内核设计逐渐成为了一个热点。微内核设计主张将传统操作系统中的设备驱动、文件系统等可在用户空间实现的功能,
移出内核,作为普通的用户程序来实现。这样,即使它们崩溃,也不会影响到整个系统的稳定。其他应用程序通过进程间通讯来请求
文件系统等相关服务。因此,在微内核中IPC是一个十分重要的机制。
\end{note}

接下来进入正题,IPC机制远远没有我们想象得那样神秘,特别是在我们这个被极度简化了的小操作系统中。
根据之前的讨论,我们能够得知这样几个细节:

\begin{itemize}
  \item IPC的目的是使两个进程之间可以通讯。
  \item IPC需要通过系统调用来实现
\end{itemize}

所谓通讯,最直观的一种理解就是交换数据。假如我们能够将让一个进程有能力将数据传递给另一个进程,
那么进程之间自然具有了相互通讯的能力。那么,要实现交换数据,我们所面临的最大的问题是什么呢?
没错,问题就在于\textbf{各个进程的地址空间是相互独立的}。相信你在实现内存管理的时候已经深刻体会到了这一点,
每个进程都有各自的地址空间,这些地址空间之间是相互独立的。因此,要想传递数据,
我们就需要想办法\textbf{把一个地址空间中的东西传给另一个地址空间}。

想要让两个完全独立的地址空间之间发生联系,最好的方式是什么呢?对,我们要去找一找它们是否存在共享的部分。
虽然地址空间本身独立,但是有些地址也许被映射到了同一物理内存上。如果你之前详细地看过进程的页表建立的部分的话,
想必你已经找到线索了。是的,线索就在env\_setup\_vm()这个函数里面。

\begin{minted}[linenos]{c}
static int
env_setup_vm(struct Env *e)
{
    //略去的无关代码
    
    for (i = PDX(UTOP); i <= PDX(~0); i++) {
        pgdir[i] = boot_pgdir[i];
    }
    e->env_pgdir = pgdir;
    e->env_cr3   = PADDR(pgdir);
    
    //略去的无关代码
}
\end{minted}

如果你之前认真思考了这个函数的话会发现,所有的进程都共享了内核所在的2G空间。对于任意的进程,这2G都是一样的。
因此,想要在不同空间之间交换数据,我们就需要借助于内核的空间来实现。那么,我们把要传递的消息放在哪里比较好呢?
恩,发送和接受消息和进程有关,消息都是由一个进程发送给另一个进程的。内核里什么地方和进程最相关呢?啊哈!进程控制块!

\begin{minted}[linenos]{c}
struct Env {
    // Lab 4 IPC
    u_int env_ipc_value;            // data value sent to us
    u_int env_ipc_from;             // envid of the sender
    u_int env_ipc_recving;          // env is blocked receiving
    u_int env_ipc_dstva;        // va at which to map received page
    u_int env_ipc_perm;     // perm of page mapping received
};
\end{minted}

果然,我们看到了我们想要的东西,env\_ipc\_value用于存放需要发给当前进程的数据。
env\_ipc\_dstva则说明了接收到的页需要被映射到哪个虚地址上。知道了这些,我们就不难实现IPC机制了。只需要做做赋值,
填充下对应的域,映射下该映射的页之类的就好了。

\begin{exercise}
实现lib/syscall\_all.c中的void sys\_ipc\_recv(int sysno,u\_int dstva)函数和
int sys\_ipc\_can\_send(int sysno,u\_int envid, u\_int value, u\_int srcva, u\_int perm)函数。
\end{exercise}

sys\_ipc\_recv(int sysno,u\_int dstva)函数首先要将env\_ipc\_recving设置为1,表明该进程准备接受其它进程的消息了。
之后阻塞当前进程,即将当前进程的状态置为不可运行。之后放弃CPU(调用相关函数重新进行调度)。

int sys\_ipc\_can\_send(int sysno,u\_int envid, u\_int value, u\_int srcva, u\_int perm)函数用于发送消息。
如果指定进程为可接收状态,则发送成功,清除接收进程的接收状态,使其可运行,返回0,否则,返会\_E\_IPC\_NOT\_RECV。

讲完IPC后,我们来利用前面已经实现了的基础系统调用来实现一个非常有意思的函数:fork。

\section{FORK}

在Lab3我们曾提到过,env\_alloc是内核产生一个进程。但如果想让一个进程创建一个进程,
就像是父亲与儿子那样,我们就需要使用到fork了。那么fork究竟是什么呢?

\subsection{初窥fork}
fork,直观意象是叉子的意思。在我们这里更像是分叉的意思,就好像一条河流动着,遇到一个分叉口,分成两条河一样,
fork就是那个分叉口。在操作系统中,在某个进程中调用fork()之后,将会以此为分叉分成两个进程运行。
新的进程在开始运行时有着和旧进程\textbf{绝大部分相同的信息},而且在新的进程中fork依旧有
一个返回值,只是该返回值为0。在旧进程,也就是所谓的父进程中,fork的返回值是子进程的env\_id,是大于0的。
在父子进程中有不同的返回值的特性,可以让我们在使用fork后很好地区分父子进程,从而安排不同的工作。

你可能会想,fork执行完为什么不直接生成一个空白的进程块,生成一个几乎和父进程一模一样的子进程有什么用呢?
换成创建一个空白的进程多简单!按笔者的理解,这是因为:
\begin{itemize}
 \item 与不相干的两个进程相比,父子进程间的通信要方便的多。因为fork虽然没法造成进程间的统治关系\footnote{这是因为进程之间是并发的,在操作系统看来,父子进程之间更像是兄弟关系。},
但是因为在子进程中记录了父进程的一些信息,父进程也可以很方便地对子进程进行一些管理等。
 \item  当然还有一个可能的原因在于安全与稳定,尤其是关于操作权限方面。对这方面有兴趣的同学可以查看链接\footnote{http://www.jbxue.com/shouce/apache2.2/mod/prefork.html}
探索一下。
\end{itemize}

fork之后父子进程就分道扬镳,互相独立了。而和fork“针锋相对”却又经常“纠缠不清”的,
是名为exec系列的系统调用。它会“勾引”子进程抛弃现有的一切,另起炉灶。若在子进程中执行exec,
完成后子进程从父进程那拷贝来的东西就全部消失了。取而代之的是一个全新的进程,就像太乙真人用莲藕
为哪吒重塑了一个肉身一样。
exec系列系统调用我们将会作为一个挑战性任务放在后面来实现,暂时不做过多介绍。

为了让你对fork的认识不只是停留在理论层面,我们下面来做一个小实验,复制到你的linux环境下运行一下吧。
\begin{codeBoxWithCaption}{理解fork\label{code:fork_test.c}}
  \inputminted[linenos]{c}{codes/fork_test.c}
\end{codeBoxWithCaption}

使用\mintinline{c}|gcc fork_test.c|,然后\mintinline{c}| ./a.out| 运行一下,你得到的正常的结果应该如下所示:
\begin{minted}[linenos]{console}
Before fork.
After fork.
After fork.
son.pid:16903 (数字不一定一样)
father.pid:16902
\end{minted}

我们从这段简短的代码里可以获取到很多的信息,比如以下几点:
\begin{itemize}
 \item 在fork之前的代码段只有父进程会执行。
 \item 在fork之后的代码段父子进程都会执行\label{fork与子进程}。
 \item fork在不同的进程中返回值不一样,在父进程中返回值不为0,在子进程中返回值为0。
 \item 父进程和子进程虽然很多信息相同,但他们的env\_id是不同的。
\end{itemize}

从上面的小实验我们也能看出来——子进程实际上就是按父进程的绝大多数信息和状态作为模板而雕琢出来的。
即使是以父进程为模板,父子进程也还是有很多不同的地方,不知细心的你从刚才的小实验中能否看出父子进程有哪些地方是明显不一样的吗?

\begin{thinking}\label{think-father-son}
 思考下面的问题,并对这两个问题谈谈你的理解:
  \begin{itemize}
   \item 子进程完全按照fork()之后父进程的代码执行,说明了什么?
   \item 但是子进程却没有执行fork()之前父进程的代码,又说明了什么?
  \end{itemize}
\end{thinking}

% \item 表明子进程拷贝了父进程的代码段。
 %\item 子进程却没有执行fork()之前父进程的代码,表明子进程开始运行时的PC值不是二进制镜像的入口!
 %\item fork有两个返回值,不是指fork在同一个进程中返回两次,而是指fork在两个进程中均有返回值,且返回值不同。

%限于笔者自身的理解与表述能力,对fork的表述还是含糊不清的,如果想获得更多的理解,推荐查看链接中的帖子:\url{http://bbs.chinaunix.net/forum.php?mod=viewthread&tid=311067}

\subsection{fork的结构}

通过使用初步了解fork后,先不着急实现它。俗话说“兵马未动,粮草先行”,我们先来了解一下关于fork的底层细节。
根据维基百科的描述,在fork时,父进程会为子进程分配独立的地址空间。但是分配独立的虚拟空间并不意味
着一定会分配额外的物理内存:父子进程用的是相同的物理空间。子进程的代码段、数据段、堆栈
都是指向父进程的物理空间,也就是说,虽然两者的虚拟空间是不同的,但是他们所对应的物理空间是同一个。

\begin{note}
\small{
Wiki Fork: In Unix systems equipped with virtual memory support (practically all modern variants), the fork operation creates a separate address space
 for the child. The child process has an exact copy of all the memory segments of the parent process, though if copy-on-write semantics 
 are implemented,the physical memory need not be actually copied. Instead, virtual memory pages in both processes may refer to the same pages of physical memory 
 until one of them writes to such a page: then it is copied. This optimization is important in the common case where fork is used 
 in conjunction with exec to execute a new program: typically, the child process performs only a small set of actions before it ceases
 execution of its program in favour of the program to be started, and it requires very few, if any, of its parent's data structures.}
\end{note}

那你可能就有问题了:既然上文提到了父子进程之间是独立的,而现在又说共享物理内存,这不是矛盾吗?
按照共享物理内存的说法, 那岂不是变成了“父教子从,子不得不从”?

这两种说法实际上不矛盾,因为父子进程共享物理内存是有前提条件的:共享的物理内存不会被任一进程修改。那么,对于那些父进程或子进程修改的内存我们又该如何处理呢?
这里我们引入一个新的概念——写时复制(copy on write,简称COW)。写时复制,通俗来讲,就是当父子进程中有\textbf{更改}相应段的行为发生时,
再为子进程相应的段分配物理空间,而\textbf{子进程的代码段继续共享父进程的物理空间}(两者的代码完全相同)。

\begin{note}
如果在fork之后在子进程中执行了exec,由于这时和父进程要执行的代码完全不同,子进程的代码段也会分配单独的物理空间。
\end{note}

更深层次地讲,COW就是父进程和子进程平时共享物理页面,写时复制物理页面。为了能够在物理页被修改时及时处理,
只要是进程可写的页面,就要\textbf{通过设置权限位PTE\_COW的方式}被保护起来。\label{页保护与处理}无论父进程还是子进程何时试图写一个被保护的
物理页,就会产生一个异常。异常发生后,操作系统检测该物理页是否为copy on write的页面, 如果是的话将原页复制一份,
然后重新映射到导致异常的虚拟地址上,之后重新执行指令。下面这张图较为生动地展示了这一过程:

\begin{figure}[htbp]
  \centering
  \includegraphics[width=12cm]{fork_cow.png}
  \caption{fork的Copy-On-Write机制}\label{fig:fork_cow} 
\end{figure}

\begin{note}
早期的Unix系统对于fork采取的策略是:直接把父进程所有的资源复制给新创建的进程。
这种实现过于简单,并且效率非常低。因为它拷贝的内存也许是需要父子进程共享的,
当然更糟的情况是,如果新进程打算通过exec执行一个新的映像,那么所有的拷贝都将前功尽弃。
\end{note}

在我们的实验中,fork呢,主要由以下几个子函数和一些我们之前填过的系统调用构成:
\begin{description}
 \item [pgfault] 还记得之前的cow吗?没错,这个函数就是为了解决这个问题而存在的。
 在我们的实验中,我们通过\textbf{pgfault}函数来对共享页保护引起的异常进行处理。
 \item [syscall\_env\_alloc] 这个系统调用是fork两个返回值的关键所在,fork是通过判断这个系统
 调用的返回值是否为0从而判断当前执行fork的是父进程还是子进程。
 \item [duppage] 这个函数就是用来给共享的物理页添加保护权限位的。在fork中我们要通过遍历用户空间
 来寻找“需要添加保护”的页——即那些可能会被进程修改的物理页。
\end{description}

\label{fork区分父子进程}
小红:“咦,不科学啊。fork的两个返回值为啥是sys\_env\_alloc的功劳?不是说\hyperref[fork与子进程]{子进程只执行fork之后的代码吗}?”

小绿:“你还别不信,还真的就是sys\_env\_alloc的功劳。我们前面是提到了子进程执行fork之后的代码,实则不准确:因为在fork内部呀,
就要用sys\_env\_alloc的两个返回值区分开父子进程,好安排他们在返回之后执行不同的任务呀!你想想,虽然子进程在被创建出来就已经有了
PCB控制块和进程上下文,但是它还缺少一个UXSTACK——用户异常栈,更重要的是,子进程是否能够开始被调度是要由父进程决定的。

我们在这里又看到了一个新的概念,异常栈。那么它和正常栈有什么区别呢?你应该知道,一般的用户进程运行时,会有自己的用户栈
(以 USTACKTOP 为栈顶)。但是当fork结束后,由于写时保护的机制,大部分可写的用户空间都被保护起来了。
所以如果之后出现了"写操作",就会使用\textbf{pgfault}来处理异常。但注意,我们是不能直接在用户栈上处理这个异常的!
因为在pgfault中会修改用户空间(比如申请一个变量,会向用户栈里写东西,要知道用户栈也被保护起来了),
如果仍然在用户空间上进行,会继续写用户空间,继续触发页异常,继续pgfault,造成死循环。
所以为了区别于用户的正常栈,内核将在另外一个栈(以 UXSTACKTOOP 为栈顶)——用户异常栈上运行用户已经向内核注册过的的异常处理程序。

在讲完上述这些后,不知道你对fork的具体流程是否弄清楚了呢?下面就让我们来具体地一个函数一个函数地攻破它吧!

\subsection{返回值的秘密}

刚刚接触fork这个函数,相信你最感兴趣的可能不是别的,而是fork的两个返回值。而我们刚才也发现,
fork的两个返回值实际上是由sys\_env\_alloc所引起的,究其根本,秘密在sys\_env\_alloc
身上。那么,我们首先来思考一个问题:

\begin{thinking}\label{think-fork的调用}
 关于fork函数的两个返回值,下面说法正确的是:
 
  A、fork在父进程中被调用两次,产生两个返回值
  
  B、fork在两个进程中分别被调用一次,产生两个不同的返回值
  
  C、fork只在父进程中被调用了一次,在两个进程中各产生一个返回值
  
  D、fork只在子进程中被调用了一次,在两个进程中各产生一个返回值
\end{thinking}

以这个问题作为我们的开题小菜,
结合刚才未让你填写的系统调用sys\_env\_alloc函数来说明fork的两个返回值。

\begin{codeBoxWithCaption}{alloc——fork之魂\label{code:sys_env_alloc.c}}
  \inputminted[linenos]{c}{codes/sys_env_alloc.c}
\end{codeBoxWithCaption}

这个系统调用就是为了新建一个进程。从系统调用的名字上来看我们也能发现,实际上这个系统调用
和\textbf{env\_alloc}的功能很像。但是我们又提到fork是一种不完全复制,所以它除了建造外,还需要
用一些当前进程的信息作为模版来填充新的进程。那么还需要复制些什么?

\begin{description}
 \item [CPU环境] 要复制一份当前进程的运行环境到子进程的env\_tf控制块里。
 \item [PC] 刚才那些知识综合一下,你应该也能推断出子进程实际上是从sys\_env\_alloc返回的地方作为起点,执行父进程的代码。
 所以子进程的PC值应该被设置为syscall\_env\_alloc返回后的地址。
 \item [返回值有关] 看到syscall\_env\_alloc提示你应该能明白,我们需要调整某个寄存器的值以让syscall\_env\_alloc在子进程中可以返回0。
 \item [进程状态] 我们当然不能让子进程在syscall\_env\_alloc返回后就直接进入调度,因为这时候它还没有做好充分的准备,所以我们需要设定不能
 让它被加入调度队列。
 \end{description}

了解到这些信息,相信你可以写出一个合格的syscall\_env\_alloc系统调用了,那么下面我们就把它填充完整。

\begin{exercise}
填写 ./lib/syscall\_all.c 中的函数 sys\_env\_alloc,可以不填返回值。
\end{exercise}

我们刚才提到了子进程好像是从sys\_env\_alloc返回之后的地方开始执行的,那么,证据呢?

证据其实离我们一点都不远——在刚才的fork内部我们讲到需要区分父子进程,所以需要判断sys\_env\_alloc的返回值,所以会执行这样结构的语句:
\begin{minted}[linenos]{c}
 envid = sys_env_alloc();
 //在父进程中
 if(envid!=0)
    ...
 //在子进程中
 else
   ...
\end{minted}

那么现在的问题就是在不同的进程中,为什么envid会有两个值,一个为0,一个非0?

实际上是这样的:在父进程运行到\mintinline{c}|envid = sys_env_alloc|这句话时,我们把它拆分成两步,其实是如下的一个过程:
\begin{minted}[linenos]{c}
#假设返回值存放的寄存器为R
sys_env_alloc() -> R
R -> envid
\end{minted}

首先运行sys\_env\_alloc()函数,它的返回值被放在了R寄存器中。下一步将R寄存器中的值赋给envid。
我们再返回来看一下前面所述\textbf{“子进程的PC值应该被设置为sys\_env\_alloc返回后的地址”。}所以当子进程开始被调度时,
子进程运行的第一条\textbf{指令}实际上是上述代码中的\mintinline{c}|R -> envid|,又因为我们之前\textbf{在父进程中已经设置子进程的进程控制块
的R寄存器值为0},所以在子进程中,envid的值是0!而在父进程中,因为父进程完整地执行完了sys\_env\_alloc,所以其R寄存器存放的正是sys\_env\_alloc的返回值。

更直观一些,请看图\ref{fig:Two_returns}:

\begin{figure}[htbp]
  \centering
  \includegraphics[width=16cm]{Two_returns.png}
  \caption{两个返回值}\label{fig:Two_returns} 
\end{figure}

\begin{exercise}
补充./lib/syscall\_all.c 中的函数 sys\_env\_alloc的返回值。
\end{exercise}

返回值的秘密我们讲完了,那么接下来该做些什么呢?当然是填写fork啦,根据\hyperref[fork区分父子进程]{上文}我们知道,
fork内部要通过\mintinline{c}|syscall_env_alloc()|的返回值来区分父子进程,那么在父进程中和在子进程中分别要完成什么任务呢?

\subsection{父与子}

fork里的\mintinline{c}| extern struct Env *env;|,这个env可是大有来历:
它是源自外部函数\textbf{./user/libos.c}里的一个全局变量。找到entry.S仔细观察一下就可以发现:在整个实验体系中,
\_start叶函数执行完毕后会跳转到libmain执行,真正的入口函数是从libmain开始的。libmain中使用env指向当前的进程以便之后进程通信的需要,
然后才开始执行我们所使用的测试函数中的umain内容。\textbf{所以,为了使得env始终指向当前进程},在子进程中我们需要进行必要的设置。

\begin{exercise}
 补充./user/fork.c 中的函数 fork 中关于sys\_env\_alloc的部分和“子进程”执行的部分。
\end{exercise}

那父进程需要做些什么呢?我们在刚才提到过了父进程的任务:父进程要帮子进程搭建一个UXSTACK,
然后为了能让这个错误栈正确地投入使用,还得帮子进程注册错误处理函数。最后父进程还需要将子进程的状态更改为RUNNABLE,
子进程就可以参与调度了,这时候父进程就可以松一口气了。

难道真的能松口气了?不,在这一切之前父进程漏了最重要的一步:没把子进程跟父进程共享的物理页保护起来。
所以,我们还需要遍历父进程的\textbf{大部分用户空间页},然后找到可写的页,让它\textbf{在父进程和子进程}中同时被PTE\_COW权限保护起来!
我们前面提到了,保护页用的函数是啥?对,就是duppage函数。

\begin{thinking}\label{think:遍历页}
	如果仔细阅读上述这一段话,你应该可以发现,我们并不是对所有的用户空间页都使用duppage进行了保护。那么究竟哪些用户空间页可以保护,哪些不可以呢,
	请结合 include/mmu.h 里的内存布局图谈谈你的看法。
\end{thinking}

\subsubsection{duppage}

在duppage函数中,唯一需要强调的一点是要对不同权限的页有着不同的处理方式。
\begin{itemize}
 \item 在父进程中可写的或者是被打上PTE\_COW标记的页,需要在子进程中为其加上PTE\_COW的标志。
 \item 在父进程中只读的页,按照相同的权限给子进程就好。
\end{itemize}

\begin{exercise}
 结合注释,补充 user/fork.c 中的函数 duppage。
\end{exercise}

\subsubsection{pgfault}
我们\hyperref[页保护与处理]{前面}提到了写时复制。pgfault就是负责处理写时异常的异常处理函数。
pgfault很简单,按下述步骤执行即可:
\begin{enumerate}
	\item 判断页是否为copy-on-write,是则进行下一步;否则报错。
	\item 分配一个新的内存页到临时位置,将要复制的内容拷贝到刚刚分配的页中;
	\item 将临时位置上的内容映射到指定地址va,然后解除临时位置对内存的映射;
\end{enumerate}

\begin{exercise}
	结合注释,补充 user/fork.c 中的函数 pgfault。 
\end{exercise}

\subsubsection{fork}
fork的填写参见上述关于父子进程各自任务的描述。其中比较难思考的一点在于\textbf{如何遍历父进程的用户空间}。在这里你需要使用vpd与vpt宏,这两个宏的用法需要你自行思考。

\begin{exercise}
结合上文描述与注释,将 user/fork.c 中的函数 fork填充完整。
\end{exercise}

\begin{thinking}\label{think:vpt的使用}
	请结合代码与示意图,回答以下两个问题:
	
	1、vpt和vpd宏的作用是什么,如何使用它们?
	
	2、它们出现的背景是什么? 如果让你在lab2中要实现同样的功能,可以怎么写?
\end{thinking}

至此,我们的lab4实验已经基本完成了,接下来就一起来愉快地调试吧!

\section{实验正确结果}

本次测试分为两个文件,当基础系统调用与fork写完后,单独测试fork的文件是\textbf{user/fktest.c},测试时将

ENV\_CREATE(user\_fktest)加入init.c即可测试。

正确结果如下:

\begin{minted}[linenos]{c}

	main.c:	main is start ...
	
	init.c:	mips_init() is called
	
	Physical memory: 65536K available, base = 65536K, extended = 0K
	
	to memory 80401000 for struct page directory.
	
	to memory 80431000 for struct Pages.
	
	mips_vm_init:boot_pgdir is 80400000
	
	pmap.c:	 mips vm init success
	
	panic at init.c:31: ^^^^^^^^^^^^^^^^^^^^^^^^^^^^^^^^^^^^^
	
	pageout:	@@@___0x7f3fe000___@@@  ins a page 
	
	this is father: a:1
	
	this is father: a:1
	
	this is father: a:1
	
	this is father: a:1
	
	this is father: a:1
	
	this is father: a:1
	
	this is father: a:1
	
	this is father: a:1
	
	this is father: a:1
	  
	  child :a:2
	
		this is child :a:2
		
		this is child :a:2
			
				this is child2 :a:3
	
				this is child2 :a:3
				
				this is child2 :a:3
								
				this is child2 :a:3
				
	this is father: a:1
	
	this is father: a:1
	
	this is father: a:1
	
	this is father: a:1
	
	this is father: a:1
	
		this is child :a:2
		
		this is child :a:2
		
		this is child :a:2
\end{minted}

另一个测试文件主要测试进程间通信,文件为\textbf{user/pingpong.c},测试方法同上。

正确结果如下:

\begin{minted}[linenos]{c}

main.c:	main is start ...

init.c:	mips_init() is called

Physical memory: 65536K available, base = 65536K, extended = 0K

to memory 80401000 for struct page directory.

to memory 80431000 for struct Pages.

mips_vm_init:boot_pgdir is 80400000

pmap.c:	 mips vm init success

panic at init.c:31: ^^^^^^^^^^^^^^^^^^^^^^^^^^^^^^^^^^^^^

pageout:	@@@___0x7f3fe000___@@@  ins a page 

pingpong is running

Fmars : parent env

Parent begins to send

1001 got 7777 from 800

@@@@@send 1 from 1001 to 800

[00001001] destroying 00001001

[00001001] free env 00001001

i am killed ... 

800 got 1 from 1001

[00000800] destroying 00000800

[00000800] free env 00000800

i am killed ... 
\end{minted}

\section{实验思考}

\begin{itemize}
	\item \hyperref[think-father-son]{\textbf{\textcolor{baseB}{思考-不同的进程代码执行}}}
	\item \hyperref[think-fork的调用]{\textbf{\textcolor{baseB}{思考-fork的返回结果}}}
	\item \hyperref[think:遍历页]{\textbf{\textcolor{baseB}{思考-用户空间的保护}}}
	\item \hyperref[think:vpt的使用]{\textbf{\textcolor{baseB}{思考-vpt宏的使用}}}
\end{itemize}

\section{挑战性任务}
我们现在实现的fork是有写时复制机制的,而sfork中是没有
\input{chapters/5-file-system}
\chapter{管道与Shell}
%目标:user/pipe.c 与 user/sh.c
%是否新增?
%是否需要自行编写测试文件?/个人觉得这里应该要求自己编写测试文件进行测试(说明白原理即可)
\section{实验目的}
\begin{enumerate}
	\item 掌握管道的原理与底层细节
	\item 实现管道的读写
	\item 复述管道竞争情景
	\item 实现shell中涉及管道的部分
\end{enumerate}

\section{管道}

在lab4中,我们已经学习过一种进程间通信(IPC,Inter-Process Communication)的方式——共享内存。
而今天我们要学的管道,其实也是进程间通信的一种方式。

\subsection{初窥管道}

通俗来讲,管道就像家里的自来水管:一端用于注入水,一端用于放出水,且水只能在一个方向上流动,而不能双向流动,所以说管道是典型的单向通信。管道又叫做匿名管道,只能用在具有公共祖先的进程之间使用,通常使用在父子进程之间通信。

在Unix中,管道由pipe函数创建,函数原型如下:

\begin{minted}[linenos]{c}
	#include<unistd.h>

	int  pipe(int fd[2]); 成功返回0,否则返回-1;

	参数fd返回两个文件描述符,fd[0]对应读端,fd[1]对应写端。
\end{minted}

为了更好地理解管道实现的原理,同样,我们先来做实验亲自体会一下\footnote{实验代码参考 http://pubs.opengroup.org/onlinepubs/9699919799/functions/pipe.html}

\begin{codeBoxWithCaption}{管道示例\label{code:test_pipe.c}}
	\inputminted[linenos]{c}{codes/test_pipe.c}
\end{codeBoxWithCaption}

示例代码实现了从父进程向子进程发送消息"Hello,world",并且在子进程中打印到屏幕上。它演示了管道在父子进程之间通信的基本用法:在pipe函数之后,调用fork来产生一个子进程,之后在父子进程中执行不同的操作。在示例代码中,父进程操作写端,而子进程操作读端。同时,示例代码也为我们演示了使用pipe系统调用的习惯:fork之后,进程在开始读或写管道之前都会关掉不会用到的管道端。

从本质上说,管道是一种只在内存中的文件。在UNIX中使用pipe系统调用时,进程中会打开两个新的文件描述符:一个只读端和一个只写端,而这两个文件描述符都映射到了同一片内存区域。但这样建立的管道的两端都在同一进程中,而且构建出的管道两端是两个匿名的文件描述符,这就让其他进程无法连接该管道。在fork的配合下,才能在父子进程间建立起进程间通信管道,这也是匿名管道只能在具有亲缘关系的进程间通信的原因。

\begin{thinking}\label{think-father-reader}
	示例代码中,父进程操作管道的写端,子进程操作管道的读端。如果现在想让父进程作为“读者”,代码应当如何修改?
\end{thinking}

\subsection{管道的测试}

我们下面就来填充函数实现匿名管道的功能。思考刚才的代码样例,要实现匿名管道,至少需要有两个功能:管道读取、管道写入。

要想实现管道,首先我们来看看本次实验我们将如何测试。lab6关于管道的测试有两个,分别是\mintinline{console}|user/testpipe.c|与\mintinline{console}|user/testpiperace.c|。

首先我们来观察testpipe的内容

\begin{codeBoxWithCaption}{testpipe测试\label{code:lab_test_pipe.c}}
	\inputminted[linenos]{c}{codes/lab_test_pipe.c}
\end{codeBoxWithCaption}

实际上可以看出,测试文件使用pipe的流程和示例代码是一致的。先使用函数 \mintinline{c}|pipe(int p[2]) |创建了管道,读端的文件控制块编号\footnote{文件控制块编号是int型,user/fd.c 中 num2fd 函数可通过它定位文件控制块的地址。}为p[0],写端的文件控制块编号为p[1]。之后使用fork()创建子进程,\textbf{注意这时父子进程使用p[0]和p[1]访问到的内存区域一致}。之后子进程关闭了p[1],从p[0]读;父进程关闭了p[0],从p[1]写入管道。

lab4的实验中,我们的fork实现是完全遵循Copy-On-Write原则的,即对于所有用户态的地址空间都进行了PTE\_COW的设置。
但实际上写时复制并不完全适用,至少在我们当前情景下是不允许写时拷贝。为什么呢?我们来看看pipe函数中的关键部分就能知晓答案:

\begin{minted}[linenos]{c}
int
pipe(int pfd[2])
{
	int r, va;
	struct Fd *fd0, *fd1;

	if ((r = fd_alloc(&fd0)) < 0
	||  (r = syscall_mem_alloc(0, (u_int)fd0, PTE_V|PTE_R|PTE_LIBRARY)) < 0)
	goto err;

	if ((r = fd_alloc(&fd1)) < 0
	||  (r = syscall_mem_alloc(0, (u_int)fd1, PTE_V|PTE_R|PTE_LIBRARY)) < 0)
	goto err1;

	va = fd2data(fd0);
	if ((r = syscall_mem_alloc(0, va, PTE_V|PTE_R|PTE_LIBRARY)) < 0)
	goto err2;
	if ((r = syscall_mem_map(0, va, 0, fd2data(fd1), PTE_V|PTE_R|PTE_LIBRARY)) < 0)
	goto err3;

	...
}
\end{minted}

在pipe中,首先分配两个文件描述符并为其分配空间,然后将一个管道作为这两个文件描述符数据区的第一页数据,从而使得这两个文件描述符能够共享一个管道的数据缓冲区。

\begin{exercise}
	仔细观察pipe中新出现的权限位\mintinline{c}|PTE_LIBRARY|,根据上述提示修改fork系统调用,使得\textbf{管道缓冲区是父子进程共享的},不设置为写时复制的模式。
\end{exercise}

下面我们使用一张图来表示父子进程与管道的数据缓冲区的关系:

\begin{figure}[htbp]
	\centering
	\includegraphics[width=15cm]{6-pipe-after-fork.eps}
	\caption{父子进程与管道缓冲区}\label{fig:6-pipe-after-fork}
\end{figure}

实际上,在父子进程中各自close掉不再使用的端口后,父子进程与管道缓冲区的关系如下图:

\begin{figure}[htbp]
	\centering
	\includegraphics[width=15cm]{6-pipe-after-close.eps}
	\caption{关闭不使用的端口后}\label{fig:6-pipe-after-close}
\end{figure}

下面我们来讲一下\mintinline{c}|struct Pipe|,并开始着手填写操作管道端的函数。

\subsection{管道的读写}

我们可以在 user/pipe.c 中轻松地找到Pipe结构体的定义,它的定义如下:

\begin{minted}[linenos]{c}
	struct Pipe {
		u_int p_rpos;		    // read position
		u_int p_wpos;		    // write position
		u_char p_buf[BY2PIPE];	// data buffer
	};
\end{minted}

在Pipe结构体中,p\_rpos给出了下一个将要从管道读的数据的位置,而p\_wpos给出了下一个将要向管道写的数据的位置。只有读者可以更新p\_rpos,同样,只有写者可以更新p\_wpos,读者和写者通过这两个变量的值进行协调读写。一个管道有BY2PIPE(32Byte)大小的缓冲区。

这个只有BY2PIPE大小的缓冲区发挥的作用类似于环形缓冲区,所以下一个要读或写的位置i实际上是i\%BY2PIPE。

读者在从管道读取数据时,要将p\_buf[p\_rpos\%BY2PIPE]的数据拷贝走,然后读指针自增1。但是需要注意的是,管道的缓冲区此时可能还没有被写入数据。所以如果管道数据为空,即当 p\_rpos >= p\_wpos时,应该进程切换到写者运行。

类似于读者,写者在向管道写入数据时,也是将数据存入p\_buf[p\_wpos\%BY2PIPE],然后写指针自增1。 需要注意管道的缓冲区可能出现满溢的情况,所以写者必须得在 p\_wpos - p\_rpos < BY2PIPE时方可运行,否则要一直挂起。

上面这些还不足以使得读者写者一定能顺利完成管道操作。假设这样的情景:管道写端已经全部关闭,读者读到缓冲区有效数据的末尾,此时有 p\_rpos = p\_wpos。按照上面的做法,我们这里应当切换到写者运行。但写者进程已经结束,进程切换就造成了死循环,这时候读者进程如何知道应当退出了呢?

为了解决上面提出的问题,我们必须得知道管道的另一端是否已经关闭。不论是在读者还是在写者中,我们都需要对另一端的状态进行判断:当出现缓冲区空或满的情况时,要根据另一端是否关闭来判断是否要返回。如果另一端已经关闭,进程返回0即可;如果没有关闭,则切换到其他进程运行。

\begin{note}
Unix : If all file descriptors referring to the write end of a pipe have been closed, then an attempt to read(2) from the pipe will see end-of-file (read(2) will return 0) link : http://linux.die.net/man/7/pipe
\end{note}

那么我们该如何知晓管道的另一端是否已经关闭了呢?这时就要用到我们的\mintinline{c}|static int _pipeisclosed(struct Fd *fd, struct Pipe *p)|函数。而这个函数的核心,就是下面我们要讲的恒成立等式了。

在之前的图\ref{fig:6-pipe-after-close}中我们没有明确画出文件描述符所占的页,但实际上,对于每一个匿名管道而言,我们分配了三页空间:一页是读数据的文件描述符rfd,一页是写数据的文件描述符wfd,剩下一页是被两个文件描述符共享的管道数据缓冲区。既然管道数据缓冲区h是被两个文件描述符所共享的,我们很直观地就能得到一个结论:如果有1个读者,1个写者,那么管道将被引用2次,就如同上图所示。pageref函数能得到页的引用次数,所以实际上有下面这个等式成立:

pageref(rfd) + pageref(wfd) = pageref(pipe)\label{variant}

\begin{note}
内核会对pages数组成员维护一个页引用变量 pp\_ref 来记录指向该物理页的虚页数量。pageref的实现实际上就是查询虚页 P 对应的实际物理页,然后返回其 pp\_ref 变量的值。
\end{note}

这个等式对我们而言有什么用呢?假设我们现在在运行读者进程,而进行管道写入的进程都已经结束了,那么此时就应该有:\mintinline{c}|pageref(wfd) = 0|。所以就有\mintinline{c}|pageref(rfd) = pageref(pipe)|。所以我们只要判断这个等式是否成绩就可以得知写端是否关闭,对写者来说同理。

\begin{exercise}
	根据上述提示与代码中的注释,填写 user/pipe.c 中的 piperead、pipewrite、\_pipeisclosed 函数并通过 testpipe 的测试。
\end{exercise}

\begin{note}
注意在本次实验中由于文件系统服务所在进程已经默认为1号进程(起始进程为0号进程),在测试时想启用文件系统需要注意ENV\_CREATE(fs\_serv) 在 init.c 中的位置。
\end{note}

\subsection{管道的竞争}

我们的小操作系统采用的是时间片轮转调度的进程调度算法,这点你应该在lab3中就深有体会了。这种抢占式的进程管理就意味着,用户进程随时有可能会被打断。

当然,如果进程间是孤立的,随时打断也没有关系。但当多个进程共享同一个变量时,执行同一段代码,不同的进程执行顺序有可能产生完全不同的结果,造成运行结果的不确定性。而进程通信需要共享(不论是管道还是共享内存),所以我们要对进程中共享变量的读写操作有足够高的警惕。

实际上,因为管道本身的共享性质,所以在管道中有一系列的竞争情况。在当前这种不加锁控制的情况下,我们无法保证\mintinline{c}|_pipeisclosed|用于管道另一端关闭的判断一定返回正确的结果。

我们重新看之前写的\mintinline{c}|_pipeisclosed|函数。在这个函数中我们对\mintinline{c}|pageref(fd structure)|与 \mintinline{c}|pageref(pipe structure)|进行了等价关系的判断。假如不考虑进程竞争,不论是在读者还是写者进程中,我们会认为:

\begin{itemize}
	\item 对fd和对pipe的pp\_ref 的\textbf{写入}是同步的。
	\item 对fd和对pipe的pp\_ref 的\textbf{读取}是同步的。
\end{itemize}

但现在我们处于进程竞争、执行顺序不定的情景下,上述两种情况现在都会出现不同步的现象。想想看,如果在下面这种场景下,我们前面提到的等式\ref{variant}还是恒成立的吗:

\label{code:example-pipe}
\begin{minted}[linenos]{c}
	pipe(p);
	if(fork() == 0 ){
		close(p[1]);
		read(p[0],buf,sizeof buf);
	}else{
		close(p[0]);
		write(p[1],"Hello",5);
	}
\end{minted}

\begin{itemize}
	\item  fork结束后,子进程先执行。时钟中断产生在close(p[1])与read之间,父进程开始执行。
	\item 父进程在close(p[0])中,p[0]已经解除了对pipe的映射(unmap),还没有来得及解除对p[0]的映射,时钟中断产生,子进程接着执行。
	\item 注意此时各个页的引用情况: pageref(p[0]) = 2(因为父进程还没有解除对p[0]的映射),而pageref(p[1]) = 1(因为子进程已经关闭了p[1])。但注意,此时pipe的pageref是2,子进程中p[0]引用了pipe,同时父进程中p[0]刚解除对pipe的映射,所以在父进程中也只有p[1]引用了pipe。
	\item 子进程执行read,read中首先判断写者是否关闭。比较pageref(pipe)与pageref(p[0])之后发现它们都是2,说明写端已经关闭,于是子进程退出。
\end{itemize}

\begin{thinking}\label{think-dup}
	上面这种不同步修改pp\_ref而导致的进程竞争问题在 user/fd.c 中的dup函数中也存在。请结合代码模仿上述情景,分析一下我们的dup函数中为什么会出现预想之外的情况?
\end{thinking}

%问题的答案也简单,实际上是在dup函数的两次map之间,正确的顺序应该是先unmap fd,再unmap pipe。

那看到这里你有可能会问:在close中,既然问题出现在两次unmap之间,那么我们为什么不能使两次unmap统一起来是一个原子操作呢?要注意,在我们的小操作系统中,只有syscall\_开头的\textbf{系统调用函数}是原子操作,其他所有包括fork这些函数%\footnote{这里我们提到fork是“库函数”,好像与Linux里对fork是系统调用的措辞是矛盾的。但笔者认为,在我们的小操作系统中fork不算是系统调用。如果你持反对意见可以在报告中提出来。}
都是可能会被打断的。一次系统调用只能unmap一页,所以我们是不能保持两次unmap为一个原子操作的。那是不是一定要两次unmap是原子操作才能使得\mintinline{c}|_pipeisclosed|一定返回正确结果呢?

\begin{thinking}\label{think-automatic}
	阅读上述材料并思考:为什么系统调用一定是原子操作呢?如果你觉得不是所有的系统调用都是原子操作,请给出反例。希望能结合相关代码进行分析。
\end{thinking}

答案当然是否定的,\mintinline{c}|_pipeisclosed|函数返回正确结果的条件其实只是:

\begin{itemize}
	\item 写端关闭 当且仅当 pageref(p[0]) == pageref(pipe);
	\item 读端关闭 当且仅当 pageref(p[1]) == pageref(pipe);
\end{itemize}

比如说第一个条件,写端关闭时,当然有pageref(p[0]) == pageref(pipe)。所以我们要解决的实际上是 \textbf{当 pageref(p[0]) == pageref(pipe) 时,写端关闭}。正面如果不好解决 问题,我们可以考虑从其逆否命题着手,即要满足:{ 当写端没有关闭的时候, pageref(p[0]) $\neq$ pageref(pipe)}。

我们考虑之前那个预想之外的情景,它出现的最关键原因在于:pipe的引用次数总比fd要高。当管道的close进行到一半时,\textbf{若先解除pipe的映射,再解除fd的映射},就会使得pipe的引用次数的-1先于fd。这就导致在两个unmap的间隙,会出现pageref(pipe) == pageref(fd)的情况。那么若调换fd和pipe在close中的unmap顺序,能否解决这个问题呢?

\begin{thinking}\label{think-race}
	仔细阅读上面这段话,并思考下列问题
	\begin{itemize}
		\item
		按照上述说法控制\textbf{pipeclose}中fd和pipe unmap的顺序,是否可以解决上述场景的进程竞争问题?给出你的分析过程。
		\item
		我们只分析了close时的情形,那么对于dup中出现的情况又该如何解决?请模仿上述材料写写你的理解。
	\end{itemize}
\end{thinking}

根据上面的描述我们其实已经能够得出一个结论:控制fd与pipe的map/unmap的顺序可以解决上述情景中出现的进程竞争问题。

那么下面根据你所思考的内容进行实践吧:

\begin{exercise}
	修改 user/pipe.c 中的 pipeclose 与 user/fd.c 中的 dup 函数 以避免上述情景中的进程竞争情况。
\end{exercise}

我们通过控制修改pp\_ref的前后顺序避免了“写数据”导致的错觉,但是我们还得解决第二个问题:
读取pp\_ref的同步问题。

同样是上面的代码\ref{code:example-pipe},我们思考下面的情景:

\begin{itemize}
	\item fork结束后,子进程先执行。执行完close(p[1])后,执行read,要从p[0]读取数据。但由于此时管道数据缓冲区为空,所以read函数要判断父进程中的写端是否关闭,进入到\_pipeisclosed函数,pageref(fd)值为2(父进程和子进程都打开了p[0]),时钟中断产生。
	\item 内核切换到父进程执行,父进程close(p[0]),之后向管道缓冲区写数据。要写的数据较多,写到一半时钟中断产生,内核切换到子进程运行。
	\item 子进程继续运行,获取到pageref(pipe)值为2(父进程打开了p[1],子进程打开了p[0]),引用值相等,于是认为父进程的写端已经关闭,子进程退出。
\end{itemize}

上述现象出现的根源在哪里呢?fd是一个父子进程共享的变量,但子进程中的pageref(fd)没有随父进程对fd的修改而同步,这就造成了子进程读到的pageref(fd)成为了“脏数据”。为了保证读的同步性,子进程应当重新读取pageref(fd)和pageref(pipe),并且要在\textbf{确认两次读取之间进程没有切换}后,才能返回正确的结果。为了实现这一点,我们要使用到之前一直都没用到的变量:env\_runs。

env\_runs记录了一个进程env\_run的次数,这样我们就可以根据某个操作do()前后进程env\_runs值是否相等,来判断在do()中进程是否发生了切换。

\begin{exercise}
	根据上面的表述,修改\mintinline{c}|_pipeisclosed|函数,使得它满足“同步读”的要求。注意env\_runs变量是需要维护的。
\end{exercise}

\section{shell}

shell本质上也是一个用户进程。它解释shell命令的工作是通过创建并运行子进程来完成的:对于每个shell命令,都有一个对应的可执行文件来完成该命令所要完成的工作,shell需要根据所得到的命令来创建执行相应可执行文件的子进程,从而完成命令的解释工作并得到结果。

为了能够使用shell,首先需要使我们的操作系统响应键盘的输入,让shell能够获得用户输入的命令。我们已经使用汇编完成了sys\_cgetc函数,你可以在 lib/getc.S 中看到它的具体实现。但是光有sys\_cgetc函数还不够,你需要增加系统调用syscall\_cgetc来获取键盘的输入。

\begin{exercise}
	模仿现有的系统调用,增加系统调用syscall\_cgetc。(提示:sys\_cgetc函数不需要传入参数)
\end{exercise}

接下来,我们需要在shell进程里实现对管道和重定向的解释功能。解释shell命令时:

\begin{enumerate}
	\item 如果碰到重定向符号‘<’或者’>’,则读下一个单词,打开这个单词所代表的文件,然后将其复制给标准输入或者标准输出。
	\item 如果碰到管道符号’|’,则首先需要建立管道pipe,然后fork。
	\begin{itemize}
		\item 对于父进程,需要将管道的写者复制给标准输出,然后关闭父进程的读者和写者,运行‘|’左边的命令,获得输出,然后等待子进程运行。
		\item 对于子进程,将管道的读者复制给标准输入,从管道中读取数据,然后关闭子进程的读者和写者,继续读下一个单词。
	\end{itemize}
\end{enumerate}

\begin{exercise}
	根据以上描述,补充完成 user/sh.c 中的 \mintinline{c}|void runcmd(char *s)|。
\end{exercise}

\section{实验正确结果}

\subsection{管道测试}
管道测试有两个文件,分别是 user/testpipe.c 和 user/testpiperace.c ,以合适的次序建好进程后,在testpipe的测试中若出现两次\textbf{pipe tests passed}即说明测试通过。在testpiperace的测试中应当出现{race didn't happen}是正确的。

\subsection{shell测试}
在 init/init.c 中按照如下顺序依次启动 shell 和 文件服务:

\begin{minted}[linenos]{c}
	ENV_CREATE(user_icode);
	ENV_CREATE(fs_serv);
\end{minted}

如果正常会看到如下现象:

\begin{minted}[linenos]{c}
	:::::::::::::::::::::::::::::::::::::::::::::::::::::::::::::

	::                                                         ::

	::              Super Shell  V0.0.0_1                      ::

	::                                                         ::

	:::::::::::::::::::::::::::::::::::::::::::::::::::::::::::::
	\$
\end{minted}

使用不同的命令会有不同的效果:
\begin{itemize}
	\item 输入ls.b,会显示一些文件和文件夹;
	\item 输入cat.b,会有回显现象出现;
	\item 输入ls.b | cat.b,和 ls.b 的现象应当一致;
\end{itemize}

\section{实验思考}

\begin{itemize}
	\item \hyperref[think-father-reader]{\textbf{\textcolor{baseB}{思考-父进程为读者}}}
	\item \hyperref[think-dup]{\textbf{\textcolor{baseB}{思考-dup中的进程竞争}}}
	\item \hyperref[think-automatic]{\textbf{\textcolor{baseB}{思考-原子操作}}}
	\item \hyperref[think-race]{\textbf{\textcolor{baseB}{思考-解决进程竞争}}}
	\item \hyperref[think-pc]{\textbf{\textcolor{baseB}{思考-进程上下文的PC值}}}
	\item \hyperref[think-TIMESTACK]{\textbf{\textcolor{baseB}{思考-TIMESTACK的含义}}}
	\item \hyperref[think-调度]{\textbf{\textcolor{baseB}{思考-不公的调度}}}
\end{itemize}





%接下来的书写思路:首先是根据testpipe.c里用到的东西来进行,然后展示pipe函数(从里面挖几个点用于),之后就是根据read和write里的面向对象形式的读写设备方式,
%开始让他们填写管道的读函数与写函数。读写直接根据注释填写即可,值得重点介绍的地方在于锁的那个地方。

\begin{appendix}
\input{chapters/0-environment}
\end{appendix}
\end{document}
